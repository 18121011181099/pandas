\documentclass[11pt]{article}

    \usepackage[breakable]{tcolorbox}
    \usepackage{parskip} % Stop auto-indenting (to mimic markdown behaviour)
    
    \usepackage{iftex}
    \ifPDFTeX
    	\usepackage[T1]{fontenc}
    	\usepackage{mathpazo}
    \else
    	\usepackage{fontspec}
    \fi

    % Basic figure setup, for now with no caption control since it's done
    % automatically by Pandoc (which extracts ![](path) syntax from Markdown).
    \usepackage{graphicx}
    % Maintain compatibility with old templates. Remove in nbconvert 6.0
    \let\Oldincludegraphics\includegraphics
    % Ensure that by default, figures have no caption (until we provide a
    % proper Figure object with a Caption API and a way to capture that
    % in the conversion process - todo).
    \usepackage{caption}
    \DeclareCaptionFormat{nocaption}{}
    \captionsetup{format=nocaption,aboveskip=0pt,belowskip=0pt}

    \usepackage{float}
    \floatplacement{figure}{H} % forces figures to be placed at the correct location
    \usepackage{xcolor} % Allow colors to be defined
    \usepackage{enumerate} % Needed for markdown enumerations to work
    \usepackage{geometry} % Used to adjust the document margins
    \usepackage{amsmath} % Equations
    \usepackage{amssymb} % Equations
    \usepackage{textcomp} % defines textquotesingle
    % Hack from http://tex.stackexchange.com/a/47451/13684:
    \AtBeginDocument{%
        \def\PYZsq{\textquotesingle}% Upright quotes in Pygmentized code
    }
    \usepackage{upquote} % Upright quotes for verbatim code
    \usepackage{eurosym} % defines \euro
    \usepackage[mathletters]{ucs} % Extended unicode (utf-8) support
    \usepackage{fancyvrb} % verbatim replacement that allows latex
    \usepackage{grffile} % extends the file name processing of package graphics 
                         % to support a larger range
    \makeatletter % fix for old versions of grffile with XeLaTeX
    \@ifpackagelater{grffile}{2019/11/01}
    {
      % Do nothing on new versions
    }
    {
      \def\Gread@@xetex#1{%
        \IfFileExists{"\Gin@base".bb}%
        {\Gread@eps{\Gin@base.bb}}%
        {\Gread@@xetex@aux#1}%
      }
    }
    \makeatother
    \usepackage[Export]{adjustbox} % Used to constrain images to a maximum size
    \adjustboxset{max size={0.9\linewidth}{0.9\paperheight}}

    % The hyperref package gives us a pdf with properly built
    % internal navigation ('pdf bookmarks' for the table of contents,
    % internal cross-reference links, web links for URLs, etc.)
    \usepackage{hyperref}
    % The default LaTeX title has an obnoxious amount of whitespace. By default,
    % titling removes some of it. It also provides customization options.
    \usepackage{titling}
    \usepackage{longtable} % longtable support required by pandoc >1.10
    \usepackage{booktabs}  % table support for pandoc > 1.12.2
    \usepackage[inline]{enumitem} % IRkernel/repr support (it uses the enumerate* environment)
    \usepackage[normalem]{ulem} % ulem is needed to support strikethroughs (\sout)
                                % normalem makes italics be italics, not underlines
    \usepackage{mathrsfs}
    

    
    % Colors for the hyperref package
    \definecolor{urlcolor}{rgb}{0,.145,.698}
    \definecolor{linkcolor}{rgb}{.71,0.21,0.01}
    \definecolor{citecolor}{rgb}{.12,.54,.11}

    % ANSI colors
    \definecolor{ansi-black}{HTML}{3E424D}
    \definecolor{ansi-black-intense}{HTML}{282C36}
    \definecolor{ansi-red}{HTML}{E75C58}
    \definecolor{ansi-red-intense}{HTML}{B22B31}
    \definecolor{ansi-green}{HTML}{00A250}
    \definecolor{ansi-green-intense}{HTML}{007427}
    \definecolor{ansi-yellow}{HTML}{DDB62B}
    \definecolor{ansi-yellow-intense}{HTML}{B27D12}
    \definecolor{ansi-blue}{HTML}{208FFB}
    \definecolor{ansi-blue-intense}{HTML}{0065CA}
    \definecolor{ansi-magenta}{HTML}{D160C4}
    \definecolor{ansi-magenta-intense}{HTML}{A03196}
    \definecolor{ansi-cyan}{HTML}{60C6C8}
    \definecolor{ansi-cyan-intense}{HTML}{258F8F}
    \definecolor{ansi-white}{HTML}{C5C1B4}
    \definecolor{ansi-white-intense}{HTML}{A1A6B2}
    \definecolor{ansi-default-inverse-fg}{HTML}{FFFFFF}
    \definecolor{ansi-default-inverse-bg}{HTML}{000000}

    % common color for the border for error outputs.
    \definecolor{outerrorbackground}{HTML}{FFDFDF}

    % commands and environments needed by pandoc snippets
    % extracted from the output of `pandoc -s`
    \providecommand{\tightlist}{%
      \setlength{\itemsep}{0pt}\setlength{\parskip}{0pt}}
    \DefineVerbatimEnvironment{Highlighting}{Verbatim}{commandchars=\\\{\}}
    % Add ',fontsize=\small' for more characters per line
    \newenvironment{Shaded}{}{}
    \newcommand{\KeywordTok}[1]{\textcolor[rgb]{0.00,0.44,0.13}{\textbf{{#1}}}}
    \newcommand{\DataTypeTok}[1]{\textcolor[rgb]{0.56,0.13,0.00}{{#1}}}
    \newcommand{\DecValTok}[1]{\textcolor[rgb]{0.25,0.63,0.44}{{#1}}}
    \newcommand{\BaseNTok}[1]{\textcolor[rgb]{0.25,0.63,0.44}{{#1}}}
    \newcommand{\FloatTok}[1]{\textcolor[rgb]{0.25,0.63,0.44}{{#1}}}
    \newcommand{\CharTok}[1]{\textcolor[rgb]{0.25,0.44,0.63}{{#1}}}
    \newcommand{\StringTok}[1]{\textcolor[rgb]{0.25,0.44,0.63}{{#1}}}
    \newcommand{\CommentTok}[1]{\textcolor[rgb]{0.38,0.63,0.69}{\textit{{#1}}}}
    \newcommand{\OtherTok}[1]{\textcolor[rgb]{0.00,0.44,0.13}{{#1}}}
    \newcommand{\AlertTok}[1]{\textcolor[rgb]{1.00,0.00,0.00}{\textbf{{#1}}}}
    \newcommand{\FunctionTok}[1]{\textcolor[rgb]{0.02,0.16,0.49}{{#1}}}
    \newcommand{\RegionMarkerTok}[1]{{#1}}
    \newcommand{\ErrorTok}[1]{\textcolor[rgb]{1.00,0.00,0.00}{\textbf{{#1}}}}
    \newcommand{\NormalTok}[1]{{#1}}
    
    % Additional commands for more recent versions of Pandoc
    \newcommand{\ConstantTok}[1]{\textcolor[rgb]{0.53,0.00,0.00}{{#1}}}
    \newcommand{\SpecialCharTok}[1]{\textcolor[rgb]{0.25,0.44,0.63}{{#1}}}
    \newcommand{\VerbatimStringTok}[1]{\textcolor[rgb]{0.25,0.44,0.63}{{#1}}}
    \newcommand{\SpecialStringTok}[1]{\textcolor[rgb]{0.73,0.40,0.53}{{#1}}}
    \newcommand{\ImportTok}[1]{{#1}}
    \newcommand{\DocumentationTok}[1]{\textcolor[rgb]{0.73,0.13,0.13}{\textit{{#1}}}}
    \newcommand{\AnnotationTok}[1]{\textcolor[rgb]{0.38,0.63,0.69}{\textbf{\textit{{#1}}}}}
    \newcommand{\CommentVarTok}[1]{\textcolor[rgb]{0.38,0.63,0.69}{\textbf{\textit{{#1}}}}}
    \newcommand{\VariableTok}[1]{\textcolor[rgb]{0.10,0.09,0.49}{{#1}}}
    \newcommand{\ControlFlowTok}[1]{\textcolor[rgb]{0.00,0.44,0.13}{\textbf{{#1}}}}
    \newcommand{\OperatorTok}[1]{\textcolor[rgb]{0.40,0.40,0.40}{{#1}}}
    \newcommand{\BuiltInTok}[1]{{#1}}
    \newcommand{\ExtensionTok}[1]{{#1}}
    \newcommand{\PreprocessorTok}[1]{\textcolor[rgb]{0.74,0.48,0.00}{{#1}}}
    \newcommand{\AttributeTok}[1]{\textcolor[rgb]{0.49,0.56,0.16}{{#1}}}
    \newcommand{\InformationTok}[1]{\textcolor[rgb]{0.38,0.63,0.69}{\textbf{\textit{{#1}}}}}
    \newcommand{\WarningTok}[1]{\textcolor[rgb]{0.38,0.63,0.69}{\textbf{\textit{{#1}}}}}
    
    
    % Define a nice break command that doesn't care if a line doesn't already
    % exist.
    \def\br{\hspace*{\fill} \\* }
    % Math Jax compatibility definitions
    \def\gt{>}
    \def\lt{<}
    \let\Oldtex\TeX
    \let\Oldlatex\LaTeX
    \renewcommand{\TeX}{\textrm{\Oldtex}}
    \renewcommand{\LaTeX}{\textrm{\Oldlatex}}
    % Document parameters
    % Document title
    \title{gabbar}
    
    
    
    
    
% Pygments definitions
\makeatletter
\def\PY@reset{\let\PY@it=\relax \let\PY@bf=\relax%
    \let\PY@ul=\relax \let\PY@tc=\relax%
    \let\PY@bc=\relax \let\PY@ff=\relax}
\def\PY@tok#1{\csname PY@tok@#1\endcsname}
\def\PY@toks#1+{\ifx\relax#1\empty\else%
    \PY@tok{#1}\expandafter\PY@toks\fi}
\def\PY@do#1{\PY@bc{\PY@tc{\PY@ul{%
    \PY@it{\PY@bf{\PY@ff{#1}}}}}}}
\def\PY#1#2{\PY@reset\PY@toks#1+\relax+\PY@do{#2}}

\expandafter\def\csname PY@tok@w\endcsname{\def\PY@tc##1{\textcolor[rgb]{0.73,0.73,0.73}{##1}}}
\expandafter\def\csname PY@tok@c\endcsname{\let\PY@it=\textit\def\PY@tc##1{\textcolor[rgb]{0.25,0.50,0.50}{##1}}}
\expandafter\def\csname PY@tok@cp\endcsname{\def\PY@tc##1{\textcolor[rgb]{0.74,0.48,0.00}{##1}}}
\expandafter\def\csname PY@tok@k\endcsname{\let\PY@bf=\textbf\def\PY@tc##1{\textcolor[rgb]{0.00,0.50,0.00}{##1}}}
\expandafter\def\csname PY@tok@kp\endcsname{\def\PY@tc##1{\textcolor[rgb]{0.00,0.50,0.00}{##1}}}
\expandafter\def\csname PY@tok@kt\endcsname{\def\PY@tc##1{\textcolor[rgb]{0.69,0.00,0.25}{##1}}}
\expandafter\def\csname PY@tok@o\endcsname{\def\PY@tc##1{\textcolor[rgb]{0.40,0.40,0.40}{##1}}}
\expandafter\def\csname PY@tok@ow\endcsname{\let\PY@bf=\textbf\def\PY@tc##1{\textcolor[rgb]{0.67,0.13,1.00}{##1}}}
\expandafter\def\csname PY@tok@nb\endcsname{\def\PY@tc##1{\textcolor[rgb]{0.00,0.50,0.00}{##1}}}
\expandafter\def\csname PY@tok@nf\endcsname{\def\PY@tc##1{\textcolor[rgb]{0.00,0.00,1.00}{##1}}}
\expandafter\def\csname PY@tok@nc\endcsname{\let\PY@bf=\textbf\def\PY@tc##1{\textcolor[rgb]{0.00,0.00,1.00}{##1}}}
\expandafter\def\csname PY@tok@nn\endcsname{\let\PY@bf=\textbf\def\PY@tc##1{\textcolor[rgb]{0.00,0.00,1.00}{##1}}}
\expandafter\def\csname PY@tok@ne\endcsname{\let\PY@bf=\textbf\def\PY@tc##1{\textcolor[rgb]{0.82,0.25,0.23}{##1}}}
\expandafter\def\csname PY@tok@nv\endcsname{\def\PY@tc##1{\textcolor[rgb]{0.10,0.09,0.49}{##1}}}
\expandafter\def\csname PY@tok@no\endcsname{\def\PY@tc##1{\textcolor[rgb]{0.53,0.00,0.00}{##1}}}
\expandafter\def\csname PY@tok@nl\endcsname{\def\PY@tc##1{\textcolor[rgb]{0.63,0.63,0.00}{##1}}}
\expandafter\def\csname PY@tok@ni\endcsname{\let\PY@bf=\textbf\def\PY@tc##1{\textcolor[rgb]{0.60,0.60,0.60}{##1}}}
\expandafter\def\csname PY@tok@na\endcsname{\def\PY@tc##1{\textcolor[rgb]{0.49,0.56,0.16}{##1}}}
\expandafter\def\csname PY@tok@nt\endcsname{\let\PY@bf=\textbf\def\PY@tc##1{\textcolor[rgb]{0.00,0.50,0.00}{##1}}}
\expandafter\def\csname PY@tok@nd\endcsname{\def\PY@tc##1{\textcolor[rgb]{0.67,0.13,1.00}{##1}}}
\expandafter\def\csname PY@tok@s\endcsname{\def\PY@tc##1{\textcolor[rgb]{0.73,0.13,0.13}{##1}}}
\expandafter\def\csname PY@tok@sd\endcsname{\let\PY@it=\textit\def\PY@tc##1{\textcolor[rgb]{0.73,0.13,0.13}{##1}}}
\expandafter\def\csname PY@tok@si\endcsname{\let\PY@bf=\textbf\def\PY@tc##1{\textcolor[rgb]{0.73,0.40,0.53}{##1}}}
\expandafter\def\csname PY@tok@se\endcsname{\let\PY@bf=\textbf\def\PY@tc##1{\textcolor[rgb]{0.73,0.40,0.13}{##1}}}
\expandafter\def\csname PY@tok@sr\endcsname{\def\PY@tc##1{\textcolor[rgb]{0.73,0.40,0.53}{##1}}}
\expandafter\def\csname PY@tok@ss\endcsname{\def\PY@tc##1{\textcolor[rgb]{0.10,0.09,0.49}{##1}}}
\expandafter\def\csname PY@tok@sx\endcsname{\def\PY@tc##1{\textcolor[rgb]{0.00,0.50,0.00}{##1}}}
\expandafter\def\csname PY@tok@m\endcsname{\def\PY@tc##1{\textcolor[rgb]{0.40,0.40,0.40}{##1}}}
\expandafter\def\csname PY@tok@gh\endcsname{\let\PY@bf=\textbf\def\PY@tc##1{\textcolor[rgb]{0.00,0.00,0.50}{##1}}}
\expandafter\def\csname PY@tok@gu\endcsname{\let\PY@bf=\textbf\def\PY@tc##1{\textcolor[rgb]{0.50,0.00,0.50}{##1}}}
\expandafter\def\csname PY@tok@gd\endcsname{\def\PY@tc##1{\textcolor[rgb]{0.63,0.00,0.00}{##1}}}
\expandafter\def\csname PY@tok@gi\endcsname{\def\PY@tc##1{\textcolor[rgb]{0.00,0.63,0.00}{##1}}}
\expandafter\def\csname PY@tok@gr\endcsname{\def\PY@tc##1{\textcolor[rgb]{1.00,0.00,0.00}{##1}}}
\expandafter\def\csname PY@tok@ge\endcsname{\let\PY@it=\textit}
\expandafter\def\csname PY@tok@gs\endcsname{\let\PY@bf=\textbf}
\expandafter\def\csname PY@tok@gp\endcsname{\let\PY@bf=\textbf\def\PY@tc##1{\textcolor[rgb]{0.00,0.00,0.50}{##1}}}
\expandafter\def\csname PY@tok@go\endcsname{\def\PY@tc##1{\textcolor[rgb]{0.53,0.53,0.53}{##1}}}
\expandafter\def\csname PY@tok@gt\endcsname{\def\PY@tc##1{\textcolor[rgb]{0.00,0.27,0.87}{##1}}}
\expandafter\def\csname PY@tok@err\endcsname{\def\PY@bc##1{\setlength{\fboxsep}{0pt}\fcolorbox[rgb]{1.00,0.00,0.00}{1,1,1}{\strut ##1}}}
\expandafter\def\csname PY@tok@kc\endcsname{\let\PY@bf=\textbf\def\PY@tc##1{\textcolor[rgb]{0.00,0.50,0.00}{##1}}}
\expandafter\def\csname PY@tok@kd\endcsname{\let\PY@bf=\textbf\def\PY@tc##1{\textcolor[rgb]{0.00,0.50,0.00}{##1}}}
\expandafter\def\csname PY@tok@kn\endcsname{\let\PY@bf=\textbf\def\PY@tc##1{\textcolor[rgb]{0.00,0.50,0.00}{##1}}}
\expandafter\def\csname PY@tok@kr\endcsname{\let\PY@bf=\textbf\def\PY@tc##1{\textcolor[rgb]{0.00,0.50,0.00}{##1}}}
\expandafter\def\csname PY@tok@bp\endcsname{\def\PY@tc##1{\textcolor[rgb]{0.00,0.50,0.00}{##1}}}
\expandafter\def\csname PY@tok@fm\endcsname{\def\PY@tc##1{\textcolor[rgb]{0.00,0.00,1.00}{##1}}}
\expandafter\def\csname PY@tok@vc\endcsname{\def\PY@tc##1{\textcolor[rgb]{0.10,0.09,0.49}{##1}}}
\expandafter\def\csname PY@tok@vg\endcsname{\def\PY@tc##1{\textcolor[rgb]{0.10,0.09,0.49}{##1}}}
\expandafter\def\csname PY@tok@vi\endcsname{\def\PY@tc##1{\textcolor[rgb]{0.10,0.09,0.49}{##1}}}
\expandafter\def\csname PY@tok@vm\endcsname{\def\PY@tc##1{\textcolor[rgb]{0.10,0.09,0.49}{##1}}}
\expandafter\def\csname PY@tok@sa\endcsname{\def\PY@tc##1{\textcolor[rgb]{0.73,0.13,0.13}{##1}}}
\expandafter\def\csname PY@tok@sb\endcsname{\def\PY@tc##1{\textcolor[rgb]{0.73,0.13,0.13}{##1}}}
\expandafter\def\csname PY@tok@sc\endcsname{\def\PY@tc##1{\textcolor[rgb]{0.73,0.13,0.13}{##1}}}
\expandafter\def\csname PY@tok@dl\endcsname{\def\PY@tc##1{\textcolor[rgb]{0.73,0.13,0.13}{##1}}}
\expandafter\def\csname PY@tok@s2\endcsname{\def\PY@tc##1{\textcolor[rgb]{0.73,0.13,0.13}{##1}}}
\expandafter\def\csname PY@tok@sh\endcsname{\def\PY@tc##1{\textcolor[rgb]{0.73,0.13,0.13}{##1}}}
\expandafter\def\csname PY@tok@s1\endcsname{\def\PY@tc##1{\textcolor[rgb]{0.73,0.13,0.13}{##1}}}
\expandafter\def\csname PY@tok@mb\endcsname{\def\PY@tc##1{\textcolor[rgb]{0.40,0.40,0.40}{##1}}}
\expandafter\def\csname PY@tok@mf\endcsname{\def\PY@tc##1{\textcolor[rgb]{0.40,0.40,0.40}{##1}}}
\expandafter\def\csname PY@tok@mh\endcsname{\def\PY@tc##1{\textcolor[rgb]{0.40,0.40,0.40}{##1}}}
\expandafter\def\csname PY@tok@mi\endcsname{\def\PY@tc##1{\textcolor[rgb]{0.40,0.40,0.40}{##1}}}
\expandafter\def\csname PY@tok@il\endcsname{\def\PY@tc##1{\textcolor[rgb]{0.40,0.40,0.40}{##1}}}
\expandafter\def\csname PY@tok@mo\endcsname{\def\PY@tc##1{\textcolor[rgb]{0.40,0.40,0.40}{##1}}}
\expandafter\def\csname PY@tok@ch\endcsname{\let\PY@it=\textit\def\PY@tc##1{\textcolor[rgb]{0.25,0.50,0.50}{##1}}}
\expandafter\def\csname PY@tok@cm\endcsname{\let\PY@it=\textit\def\PY@tc##1{\textcolor[rgb]{0.25,0.50,0.50}{##1}}}
\expandafter\def\csname PY@tok@cpf\endcsname{\let\PY@it=\textit\def\PY@tc##1{\textcolor[rgb]{0.25,0.50,0.50}{##1}}}
\expandafter\def\csname PY@tok@c1\endcsname{\let\PY@it=\textit\def\PY@tc##1{\textcolor[rgb]{0.25,0.50,0.50}{##1}}}
\expandafter\def\csname PY@tok@cs\endcsname{\let\PY@it=\textit\def\PY@tc##1{\textcolor[rgb]{0.25,0.50,0.50}{##1}}}

\def\PYZbs{\char`\\}
\def\PYZus{\char`\_}
\def\PYZob{\char`\{}
\def\PYZcb{\char`\}}
\def\PYZca{\char`\^}
\def\PYZam{\char`\&}
\def\PYZlt{\char`\<}
\def\PYZgt{\char`\>}
\def\PYZsh{\char`\#}
\def\PYZpc{\char`\%}
\def\PYZdl{\char`\$}
\def\PYZhy{\char`\-}
\def\PYZsq{\char`\'}
\def\PYZdq{\char`\"}
\def\PYZti{\char`\~}
% for compatibility with earlier versions
\def\PYZat{@}
\def\PYZlb{[}
\def\PYZrb{]}
\makeatother


    % For linebreaks inside Verbatim environment from package fancyvrb. 
    \makeatletter
        \newbox\Wrappedcontinuationbox 
        \newbox\Wrappedvisiblespacebox 
        \newcommand*\Wrappedvisiblespace {\textcolor{red}{\textvisiblespace}} 
        \newcommand*\Wrappedcontinuationsymbol {\textcolor{red}{\llap{\tiny$\m@th\hookrightarrow$}}} 
        \newcommand*\Wrappedcontinuationindent {3ex } 
        \newcommand*\Wrappedafterbreak {\kern\Wrappedcontinuationindent\copy\Wrappedcontinuationbox} 
        % Take advantage of the already applied Pygments mark-up to insert 
        % potential linebreaks for TeX processing. 
        %        {, <, #, %, $, ' and ": go to next line. 
        %        _, }, ^, &, >, - and ~: stay at end of broken line. 
        % Use of \textquotesingle for straight quote. 
        \newcommand*\Wrappedbreaksatspecials {% 
            \def\PYGZus{\discretionary{\char`\_}{\Wrappedafterbreak}{\char`\_}}% 
            \def\PYGZob{\discretionary{}{\Wrappedafterbreak\char`\{}{\char`\{}}% 
            \def\PYGZcb{\discretionary{\char`\}}{\Wrappedafterbreak}{\char`\}}}% 
            \def\PYGZca{\discretionary{\char`\^}{\Wrappedafterbreak}{\char`\^}}% 
            \def\PYGZam{\discretionary{\char`\&}{\Wrappedafterbreak}{\char`\&}}% 
            \def\PYGZlt{\discretionary{}{\Wrappedafterbreak\char`\<}{\char`\<}}% 
            \def\PYGZgt{\discretionary{\char`\>}{\Wrappedafterbreak}{\char`\>}}% 
            \def\PYGZsh{\discretionary{}{\Wrappedafterbreak\char`\#}{\char`\#}}% 
            \def\PYGZpc{\discretionary{}{\Wrappedafterbreak\char`\%}{\char`\%}}% 
            \def\PYGZdl{\discretionary{}{\Wrappedafterbreak\char`\$}{\char`\$}}% 
            \def\PYGZhy{\discretionary{\char`\-}{\Wrappedafterbreak}{\char`\-}}% 
            \def\PYGZsq{\discretionary{}{\Wrappedafterbreak\textquotesingle}{\textquotesingle}}% 
            \def\PYGZdq{\discretionary{}{\Wrappedafterbreak\char`\"}{\char`\"}}% 
            \def\PYGZti{\discretionary{\char`\~}{\Wrappedafterbreak}{\char`\~}}% 
        } 
        % Some characters . , ; ? ! / are not pygmentized. 
        % This macro makes them "active" and they will insert potential linebreaks 
        \newcommand*\Wrappedbreaksatpunct {% 
            \lccode`\~`\.\lowercase{\def~}{\discretionary{\hbox{\char`\.}}{\Wrappedafterbreak}{\hbox{\char`\.}}}% 
            \lccode`\~`\,\lowercase{\def~}{\discretionary{\hbox{\char`\,}}{\Wrappedafterbreak}{\hbox{\char`\,}}}% 
            \lccode`\~`\;\lowercase{\def~}{\discretionary{\hbox{\char`\;}}{\Wrappedafterbreak}{\hbox{\char`\;}}}% 
            \lccode`\~`\:\lowercase{\def~}{\discretionary{\hbox{\char`\:}}{\Wrappedafterbreak}{\hbox{\char`\:}}}% 
            \lccode`\~`\?\lowercase{\def~}{\discretionary{\hbox{\char`\?}}{\Wrappedafterbreak}{\hbox{\char`\?}}}% 
            \lccode`\~`\!\lowercase{\def~}{\discretionary{\hbox{\char`\!}}{\Wrappedafterbreak}{\hbox{\char`\!}}}% 
            \lccode`\~`\/\lowercase{\def~}{\discretionary{\hbox{\char`\/}}{\Wrappedafterbreak}{\hbox{\char`\/}}}% 
            \catcode`\.\active
            \catcode`\,\active 
            \catcode`\;\active
            \catcode`\:\active
            \catcode`\?\active
            \catcode`\!\active
            \catcode`\/\active 
            \lccode`\~`\~ 	
        }
    \makeatother

    \let\OriginalVerbatim=\Verbatim
    \makeatletter
    \renewcommand{\Verbatim}[1][1]{%
        %\parskip\z@skip
        \sbox\Wrappedcontinuationbox {\Wrappedcontinuationsymbol}%
        \sbox\Wrappedvisiblespacebox {\FV@SetupFont\Wrappedvisiblespace}%
        \def\FancyVerbFormatLine ##1{\hsize\linewidth
            \vtop{\raggedright\hyphenpenalty\z@\exhyphenpenalty\z@
                \doublehyphendemerits\z@\finalhyphendemerits\z@
                \strut ##1\strut}%
        }%
        % If the linebreak is at a space, the latter will be displayed as visible
        % space at end of first line, and a continuation symbol starts next line.
        % Stretch/shrink are however usually zero for typewriter font.
        \def\FV@Space {%
            \nobreak\hskip\z@ plus\fontdimen3\font minus\fontdimen4\font
            \discretionary{\copy\Wrappedvisiblespacebox}{\Wrappedafterbreak}
            {\kern\fontdimen2\font}%
        }%
        
        % Allow breaks at special characters using \PYG... macros.
        \Wrappedbreaksatspecials
        % Breaks at punctuation characters . , ; ? ! and / need catcode=\active 	
        \OriginalVerbatim[#1,codes*=\Wrappedbreaksatpunct]%
    }
    \makeatother

    % Exact colors from NB
    \definecolor{incolor}{HTML}{303F9F}
    \definecolor{outcolor}{HTML}{D84315}
    \definecolor{cellborder}{HTML}{CFCFCF}
    \definecolor{cellbackground}{HTML}{F7F7F7}
    
    % prompt
    \makeatletter
    \newcommand{\boxspacing}{\kern\kvtcb@left@rule\kern\kvtcb@boxsep}
    \makeatother
    \newcommand{\prompt}[4]{
        {\ttfamily\llap{{\color{#2}[#3]:\hspace{3pt}#4}}\vspace{-\baselineskip}}
    }
    

    
    % Prevent overflowing lines due to hard-to-break entities
    \sloppy 
    % Setup hyperref package
    \hypersetup{
      breaklinks=true,  % so long urls are correctly broken across lines
      colorlinks=true,
      urlcolor=urlcolor,
      linkcolor=linkcolor,
      citecolor=citecolor,
      }
    % Slightly bigger margins than the latex defaults
    
    \geometry{verbose,tmargin=1in,bmargin=1in,lmargin=1in,rmargin=1in}
    
    

\begin{document}
    
    \maketitle
    
    

    
    \begin{tcolorbox}[breakable, size=fbox, boxrule=1pt, pad at break*=1mm,colback=cellbackground, colframe=cellborder]
\prompt{In}{incolor}{1}{\boxspacing}
\begin{Verbatim}[commandchars=\\\{\}]
\PY{k+kn}{import} \PY{n+nn}{pandas} \PY{k}{as} \PY{n+nn}{pd}
\PY{k+kn}{import} \PY{n+nn}{numpy} \PY{k}{as} \PY{n+nn}{np}
\end{Verbatim}
\end{tcolorbox}

    \begin{tcolorbox}[breakable, size=fbox, boxrule=1pt, pad at break*=1mm,colback=cellbackground, colframe=cellborder]
\prompt{In}{incolor}{2}{\boxspacing}
\begin{Verbatim}[commandchars=\\\{\}]
\PY{n}{dict3} \PY{o}{=} \PY{p}{\PYZob{}} 
\PY{l+s+s2}{\PYZdq{}}\PY{l+s+s2}{name}\PY{l+s+s2}{\PYZdq{}}\PY{p}{:}\PY{p}{[}\PY{l+s+s1}{\PYZsq{}}\PY{l+s+s1}{ss}\PY{l+s+s1}{\PYZsq{}}\PY{p}{,}\PY{l+s+s1}{\PYZsq{}}\PY{l+s+s1}{dd}\PY{l+s+s1}{\PYZsq{}}\PY{p}{,}\PY{l+s+s1}{\PYZsq{}}\PY{l+s+s1}{rr}\PY{l+s+s1}{\PYZsq{}}\PY{p}{,}\PY{l+s+s1}{\PYZsq{}}\PY{l+s+s1}{cc}\PY{l+s+s1}{\PYZsq{}}\PY{p}{,}\PY{l+s+s1}{\PYZsq{}}\PY{l+s+s1}{ff}\PY{l+s+s1}{\PYZsq{}}\PY{p}{]} \PY{p}{,} \PY{l+s+s2}{\PYZdq{}}\PY{l+s+s2}{sub}\PY{l+s+s2}{\PYZdq{}}\PY{p}{:}\PY{p}{[}\PY{l+m+mi}{2}\PY{p}{,}\PY{l+m+mi}{4}\PY{p}{,}\PY{l+m+mi}{5}\PY{p}{,}\PY{l+m+mi}{7}\PY{p}{,}\PY{l+m+mi}{8}\PY{p}{]} \PY{p}{,} \PY{l+s+s2}{\PYZdq{}}\PY{l+s+s2}{age}\PY{l+s+s2}{\PYZdq{}}\PY{p}{:}\PY{p}{[}\PY{l+m+mi}{23}\PY{p}{,}\PY{l+m+mi}{45}\PY{p}{,}\PY{l+m+mi}{65}\PY{p}{,}\PY{l+m+mi}{25}\PY{p}{,}\PY{l+m+mi}{87}\PY{p}{]} \PY{p}{,} \PY{l+s+s2}{\PYZdq{}}\PY{l+s+s2}{weight}\PY{l+s+s2}{\PYZdq{}}\PY{p}{:}\PY{p}{[}\PY{l+m+mi}{45}\PY{p}{,}\PY{l+m+mi}{56}\PY{p}{,}\PY{l+m+mi}{76}\PY{p}{,}\PY{l+m+mi}{100}\PY{p}{,}\PY{l+m+mi}{459}\PY{p}{]}
 \PY{p}{\PYZcb{}}
\end{Verbatim}
\end{tcolorbox}

    \begin{tcolorbox}[breakable, size=fbox, boxrule=1pt, pad at break*=1mm,colback=cellbackground, colframe=cellborder]
\prompt{In}{incolor}{3}{\boxspacing}
\begin{Verbatim}[commandchars=\\\{\}]
\PY{n}{df} \PY{o}{=} \PY{n}{pd}\PY{o}{.}\PY{n}{DataFrame}\PY{p}{(}\PY{n}{dict3}\PY{p}{)}
\end{Verbatim}
\end{tcolorbox}

    \begin{tcolorbox}[breakable, size=fbox, boxrule=1pt, pad at break*=1mm,colback=cellbackground, colframe=cellborder]
\prompt{In}{incolor}{4}{\boxspacing}
\begin{Verbatim}[commandchars=\\\{\}]
\PY{n}{df}
\end{Verbatim}
\end{tcolorbox}

            \begin{tcolorbox}[breakable, size=fbox, boxrule=.5pt, pad at break*=1mm, opacityfill=0]
\prompt{Out}{outcolor}{4}{\boxspacing}
\begin{Verbatim}[commandchars=\\\{\}]
  name  sub  age  weight
0   ss    2   23      45
1   dd    4   45      56
2   rr    5   65      76
3   cc    7   25     100
4   ff    8   87     459
\end{Verbatim}
\end{tcolorbox}
        
    \begin{tcolorbox}[breakable, size=fbox, boxrule=1pt, pad at break*=1mm,colback=cellbackground, colframe=cellborder]
\prompt{In}{incolor}{5}{\boxspacing}
\begin{Verbatim}[commandchars=\\\{\}]
\PY{n}{df}\PY{o}{.}\PY{n}{to\PYZus{}csv}\PY{p}{(}\PY{l+s+s1}{\PYZsq{}}\PY{l+s+s1}{name.csv}\PY{l+s+s1}{\PYZsq{}}\PY{p}{)}
\end{Verbatim}
\end{tcolorbox}

    \begin{tcolorbox}[breakable, size=fbox, boxrule=1pt, pad at break*=1mm,colback=cellbackground, colframe=cellborder]
\prompt{In}{incolor}{6}{\boxspacing}
\begin{Verbatim}[commandchars=\\\{\}]
 \PY{n}{df}\PY{o}{.}\PY{n}{head}\PY{p}{(}\PY{l+m+mi}{3}\PY{p}{)}
\end{Verbatim}
\end{tcolorbox}

            \begin{tcolorbox}[breakable, size=fbox, boxrule=.5pt, pad at break*=1mm, opacityfill=0]
\prompt{Out}{outcolor}{6}{\boxspacing}
\begin{Verbatim}[commandchars=\\\{\}]
  name  sub  age  weight
0   ss    2   23      45
1   dd    4   45      56
2   rr    5   65      76
\end{Verbatim}
\end{tcolorbox}
        
    \begin{tcolorbox}[breakable, size=fbox, boxrule=1pt, pad at break*=1mm,colback=cellbackground, colframe=cellborder]
\prompt{In}{incolor}{7}{\boxspacing}
\begin{Verbatim}[commandchars=\\\{\}]
\PY{n}{df}\PY{o}{.}\PY{n}{tail}\PY{p}{(}\PY{l+m+mi}{3}\PY{p}{)}
\end{Verbatim}
\end{tcolorbox}

            \begin{tcolorbox}[breakable, size=fbox, boxrule=.5pt, pad at break*=1mm, opacityfill=0]
\prompt{Out}{outcolor}{7}{\boxspacing}
\begin{Verbatim}[commandchars=\\\{\}]
  name  sub  age  weight
2   rr    5   65      76
3   cc    7   25     100
4   ff    8   87     459
\end{Verbatim}
\end{tcolorbox}
        
    \begin{tcolorbox}[breakable, size=fbox, boxrule=1pt, pad at break*=1mm,colback=cellbackground, colframe=cellborder]
\prompt{In}{incolor}{8}{\boxspacing}
\begin{Verbatim}[commandchars=\\\{\}]
\PY{n}{df}\PY{o}{.}\PY{n}{describe}\PY{p}{(}\PY{p}{)}
\end{Verbatim}
\end{tcolorbox}

            \begin{tcolorbox}[breakable, size=fbox, boxrule=.5pt, pad at break*=1mm, opacityfill=0]
\prompt{Out}{outcolor}{8}{\boxspacing}
\begin{Verbatim}[commandchars=\\\{\}]
            sub        age      weight
count  5.000000   5.000000    5.000000
mean   5.200000  49.000000  147.200000
std    2.387467  27.239677  175.555404
min    2.000000  23.000000   45.000000
25\%    4.000000  25.000000   56.000000
50\%    5.000000  45.000000   76.000000
75\%    7.000000  65.000000  100.000000
max    8.000000  87.000000  459.000000
\end{Verbatim}
\end{tcolorbox}
        
    \begin{tcolorbox}[breakable, size=fbox, boxrule=1pt, pad at break*=1mm,colback=cellbackground, colframe=cellborder]
\prompt{In}{incolor}{9}{\boxspacing}
\begin{Verbatim}[commandchars=\\\{\}]
\PY{n}{dict3}\PY{p}{[}\PY{l+s+s1}{\PYZsq{}}\PY{l+s+s1}{sub}\PY{l+s+s1}{\PYZsq{}}\PY{p}{]}\PY{p}{[}\PY{l+m+mi}{1}\PY{p}{]}\PY{o}{=}\PY{l+m+mi}{17}
\end{Verbatim}
\end{tcolorbox}

    \begin{tcolorbox}[breakable, size=fbox, boxrule=1pt, pad at break*=1mm,colback=cellbackground, colframe=cellborder]
\prompt{In}{incolor}{10}{\boxspacing}
\begin{Verbatim}[commandchars=\\\{\}]
\PY{n}{dict3}\PY{p}{[}\PY{l+s+s1}{\PYZsq{}}\PY{l+s+s1}{sub}\PY{l+s+s1}{\PYZsq{}}\PY{p}{]}\PY{p}{[}\PY{l+m+mi}{2}\PY{p}{]}
\end{Verbatim}
\end{tcolorbox}

            \begin{tcolorbox}[breakable, size=fbox, boxrule=.5pt, pad at break*=1mm, opacityfill=0]
\prompt{Out}{outcolor}{10}{\boxspacing}
\begin{Verbatim}[commandchars=\\\{\}]
5
\end{Verbatim}
\end{tcolorbox}
        
    \begin{tcolorbox}[breakable, size=fbox, boxrule=1pt, pad at break*=1mm,colback=cellbackground, colframe=cellborder]
\prompt{In}{incolor}{11}{\boxspacing}
\begin{Verbatim}[commandchars=\\\{\}]
\PY{n}{dict3}\PY{o}{.}\PY{n}{index} \PY{o}{=} \PY{p}{[}\PY{l+s+s1}{\PYZsq{}}\PY{l+s+s1}{5}\PY{l+s+s1}{\PYZsq{}}\PY{p}{,}\PY{l+s+s1}{\PYZsq{}}\PY{l+s+s1}{ff}\PY{l+s+s1}{\PYZsq{}}\PY{p}{,}\PY{l+s+s1}{\PYZsq{}}\PY{l+s+s1}{4}\PY{l+s+s1}{\PYZsq{}}\PY{p}{,}\PY{l+s+s1}{\PYZsq{}}\PY{l+s+s1}{7}\PY{l+s+s1}{\PYZsq{}}\PY{p}{,}\PY{l+s+s1}{\PYZsq{}}\PY{l+s+s1}{dd}\PY{l+s+s1}{\PYZsq{}}\PY{p}{]}
\end{Verbatim}
\end{tcolorbox}

    \begin{Verbatim}[commandchars=\\\{\}, frame=single, framerule=2mm, rulecolor=\color{outerrorbackground}]
\textcolor{ansi-red-intense}{\textbf{---------------------------------------------------------------------------}}
\textcolor{ansi-red-intense}{\textbf{AttributeError}}                            Traceback (most recent call last)
\textcolor{ansi-green-intense}{\textbf{<ipython-input-11-f6c7ce4223f3>}} in \textcolor{ansi-cyan}{<module>}
\textcolor{ansi-green-intense}{\textbf{----> 1}}\textcolor{ansi-yellow-intense}{\textbf{ }}dict3\textcolor{ansi-yellow-intense}{\textbf{.}}index \textcolor{ansi-yellow-intense}{\textbf{=}} \textcolor{ansi-yellow-intense}{\textbf{[}}\textcolor{ansi-blue-intense}{\textbf{'5'}}\textcolor{ansi-yellow-intense}{\textbf{,}}\textcolor{ansi-blue-intense}{\textbf{'ff'}}\textcolor{ansi-yellow-intense}{\textbf{,}}\textcolor{ansi-blue-intense}{\textbf{'4'}}\textcolor{ansi-yellow-intense}{\textbf{,}}\textcolor{ansi-blue-intense}{\textbf{'7'}}\textcolor{ansi-yellow-intense}{\textbf{,}}\textcolor{ansi-blue-intense}{\textbf{'dd'}}\textcolor{ansi-yellow-intense}{\textbf{]}}

\textcolor{ansi-red-intense}{\textbf{AttributeError}}: 'dict' object has no attribute 'index'
    \end{Verbatim}

    \begin{tcolorbox}[breakable, size=fbox, boxrule=1pt, pad at break*=1mm,colback=cellbackground, colframe=cellborder]
\prompt{In}{incolor}{12}{\boxspacing}
\begin{Verbatim}[commandchars=\\\{\}]
\PY{n}{shivam} \PY{o}{=} \PY{n}{pd}\PY{o}{.}\PY{n}{read\PYZus{}csv}\PY{p}{(}\PY{l+s+s1}{\PYZsq{}}\PY{l+s+s1}{marks.csv}\PY{l+s+s1}{\PYZsq{}}\PY{p}{)} \PY{c+c1}{\PYZsh{}\PYZsh{}\PYZsh{}excel file ko notebook me bulna ho to ye fun ka use karte\PYZsh{}\PYZsh{}\PYZsh{}}
\end{Verbatim}
\end{tcolorbox}

    \begin{tcolorbox}[breakable, size=fbox, boxrule=1pt, pad at break*=1mm,colback=cellbackground, colframe=cellborder]
\prompt{In}{incolor}{13}{\boxspacing}
\begin{Verbatim}[commandchars=\\\{\}]
\PY{n}{shivam} \PY{c+c1}{\PYZsh{}\PYZsh{} deko a gaya\PYZsh{}\PYZsh{}}
\end{Verbatim}
\end{tcolorbox}

            \begin{tcolorbox}[breakable, size=fbox, boxrule=.5pt, pad at break*=1mm, opacityfill=0]
\prompt{Out}{outcolor}{13}{\boxspacing}
\begin{Verbatim}[commandchars=\\\{\}]
   Unnamed: 0     name  marks      city
0           0   shivam     90    raipur
1           1  shubham     78  jabalpur
2           2    suraj     87    bhopal
3           3   shymul     67   shahdol
\end{Verbatim}
\end{tcolorbox}
        
    \begin{tcolorbox}[breakable, size=fbox, boxrule=1pt, pad at break*=1mm,colback=cellbackground, colframe=cellborder]
\prompt{In}{incolor}{14}{\boxspacing}
\begin{Verbatim}[commandchars=\\\{\}]
\PY{n}{shivam}\PY{o}{.}\PY{n}{index} \PY{o}{=} \PY{p}{[}\PY{l+s+s1}{\PYZsq{}}\PY{l+s+s1}{5}\PY{l+s+s1}{\PYZsq{}}\PY{p}{,}\PY{l+s+s1}{\PYZsq{}}\PY{l+s+s1}{ff}\PY{l+s+s1}{\PYZsq{}}\PY{p}{,}\PY{l+s+s1}{\PYZsq{}}\PY{l+s+s1}{4}\PY{l+s+s1}{\PYZsq{}}\PY{p}{,}\PY{l+s+s1}{\PYZsq{}}\PY{l+s+s1}{7}\PY{l+s+s1}{\PYZsq{}}\PY{p}{]} \PY{c+c1}{\PYZsh{} index ko chnage kar rahe \PYZsh{}}
\end{Verbatim}
\end{tcolorbox}

    \begin{tcolorbox}[breakable, size=fbox, boxrule=1pt, pad at break*=1mm,colback=cellbackground, colframe=cellborder]
\prompt{In}{incolor}{15}{\boxspacing}
\begin{Verbatim}[commandchars=\\\{\}]
\PY{n}{shivam} \PY{c+c1}{\PYZsh{} deko index vale change ho gye hai op \PYZsh{}}
\end{Verbatim}
\end{tcolorbox}

            \begin{tcolorbox}[breakable, size=fbox, boxrule=.5pt, pad at break*=1mm, opacityfill=0]
\prompt{Out}{outcolor}{15}{\boxspacing}
\begin{Verbatim}[commandchars=\\\{\}]
    Unnamed: 0     name  marks      city
5            0   shivam     90    raipur
ff           1  shubham     78  jabalpur
4            2    suraj     87    bhopal
7            3   shymul     67   shahdol
\end{Verbatim}
\end{tcolorbox}
        
    \begin{tcolorbox}[breakable, size=fbox, boxrule=1pt, pad at break*=1mm,colback=cellbackground, colframe=cellborder]
\prompt{In}{incolor}{16}{\boxspacing}
\begin{Verbatim}[commandchars=\\\{\}]
\PY{n}{shivam}\PY{p}{[}\PY{l+s+s1}{\PYZsq{}}\PY{l+s+s1}{city}\PY{l+s+s1}{\PYZsq{}}\PY{p}{]}\PY{p}{[}\PY{l+m+mi}{2}\PY{p}{]} \PY{c+c1}{\PYZsh{} coloum ke value ko bula rahe hai\PYZsh{} }
\end{Verbatim}
\end{tcolorbox}

            \begin{tcolorbox}[breakable, size=fbox, boxrule=.5pt, pad at break*=1mm, opacityfill=0]
\prompt{Out}{outcolor}{16}{\boxspacing}
\begin{Verbatim}[commandchars=\\\{\}]
'bhopal'
\end{Verbatim}
\end{tcolorbox}
        
    \begin{tcolorbox}[breakable, size=fbox, boxrule=1pt, pad at break*=1mm,colback=cellbackground, colframe=cellborder]
\prompt{In}{incolor}{17}{\boxspacing}
\begin{Verbatim}[commandchars=\\\{\}]
\PY{n}{shivam} \PY{p}{[}\PY{l+s+s1}{\PYZsq{}}\PY{l+s+s1}{city}\PY{l+s+s1}{\PYZsq{}}\PY{p}{]}\PY{p}{[}\PY{l+m+mi}{2}\PY{p}{]} \PY{o}{=}\PY{l+s+s1}{\PYZsq{}}\PY{l+s+s1}{indore}\PY{l+s+s1}{\PYZsq{}} \PY{c+c1}{\PYZsh{} coloum ke vale ko change kar rahe hai index kke maded se \PYZsh{}}
\end{Verbatim}
\end{tcolorbox}

    \begin{Verbatim}[commandchars=\\\{\}]
<ipython-input-17-3548167da00b>:1: SettingWithCopyWarning:
A value is trying to be set on a copy of a slice from a DataFrame

See the caveats in the documentation: https://pandas.pydata.org/pandas-
docs/stable/user\_guide/indexing.html\#returning-a-view-versus-a-copy
  shivam ['city'][2] ='indore' \# coloum ke vale ko change kar rahe hai index kke
maded se \#
    \end{Verbatim}

    \begin{tcolorbox}[breakable, size=fbox, boxrule=1pt, pad at break*=1mm,colback=cellbackground, colframe=cellborder]
\prompt{In}{incolor}{18}{\boxspacing}
\begin{Verbatim}[commandchars=\\\{\}]
\PY{n}{shivam} \PY{c+c1}{\PYZsh{} deko ho gaya channge \PYZsh{}}
\end{Verbatim}
\end{tcolorbox}

            \begin{tcolorbox}[breakable, size=fbox, boxrule=.5pt, pad at break*=1mm, opacityfill=0]
\prompt{Out}{outcolor}{18}{\boxspacing}
\begin{Verbatim}[commandchars=\\\{\}]
    Unnamed: 0     name  marks      city
5            0   shivam     90    raipur
ff           1  shubham     78  jabalpur
4            2    suraj     87    indore
7            3   shymul     67   shahdol
\end{Verbatim}
\end{tcolorbox}
        
    \begin{tcolorbox}[breakable, size=fbox, boxrule=1pt, pad at break*=1mm,colback=cellbackground, colframe=cellborder]
\prompt{In}{incolor}{19}{\boxspacing}
\begin{Verbatim}[commandchars=\\\{\}]
\PY{n}{shivam}\PY{o}{.}\PY{n}{to\PYZus{}csv}\PY{p}{(}\PY{l+s+s1}{\PYZsq{}}\PY{l+s+s1}{shivam2\PYZus{}csv}\PY{l+s+s1}{\PYZsq{}}\PY{p}{)} \PY{c+c1}{\PYZsh{} csv file wapas se save  ho gaya hai \PYZsh{} }
\end{Verbatim}
\end{tcolorbox}

    \begin{tcolorbox}[breakable, size=fbox, boxrule=1pt, pad at break*=1mm,colback=cellbackground, colframe=cellborder]
\prompt{In}{incolor}{20}{\boxspacing}
\begin{Verbatim}[commandchars=\\\{\}]
\PY{c+c1}{\PYZsh{}\PYZsh{} chaliye shuru karte hai \PYZsh{}\PYZsh{}}
\end{Verbatim}
\end{tcolorbox}

    \begin{tcolorbox}[breakable, size=fbox, boxrule=1pt, pad at break*=1mm,colback=cellbackground, colframe=cellborder]
\prompt{In}{incolor}{21}{\boxspacing}
\begin{Verbatim}[commandchars=\\\{\}]
\PY{n}{newdfff}  \PY{o}{=} \PY{n}{pd}\PY{o}{.}\PY{n}{DataFrame}\PY{p}{(}\PY{n}{np}\PY{o}{.}\PY{n}{random}\PY{o}{.}\PY{n}{rand}\PY{p}{(}\PY{l+m+mi}{334}\PY{p}{,}\PY{l+m+mi}{5}\PY{p}{)} \PY{p}{,} \PY{n}{index} \PY{o}{=} \PY{n}{np}\PY{o}{.}\PY{n}{arange}\PY{p}{(}\PY{l+m+mi}{334}\PY{p}{)}\PY{p}{)}\PY{c+c1}{\PYZsh{}\PYZsh{} creation of dataframe \PYZsh{}\PYZsh{}}
\end{Verbatim}
\end{tcolorbox}

    \begin{tcolorbox}[breakable, size=fbox, boxrule=1pt, pad at break*=1mm,colback=cellbackground, colframe=cellborder]
\prompt{In}{incolor}{22}{\boxspacing}
\begin{Verbatim}[commandchars=\\\{\}]
\PY{n}{newdfff} \PY{c+c1}{\PYZsh{} calling new dataframe\PYZsh{}}
\end{Verbatim}
\end{tcolorbox}

            \begin{tcolorbox}[breakable, size=fbox, boxrule=.5pt, pad at break*=1mm, opacityfill=0]
\prompt{Out}{outcolor}{22}{\boxspacing}
\begin{Verbatim}[commandchars=\\\{\}]
            0         1         2         3         4
0    0.781981  0.160657  0.306927  0.597240  0.817418
1    0.337682  0.032490  0.578410  0.196446  0.459194
2    0.999539  0.553314  0.687496  0.716025  0.397865
3    0.092922  0.507143  0.230202  0.512950  0.298372
4    0.658756  0.129097  0.090615  0.637531  0.647706
..        {\ldots}       {\ldots}       {\ldots}       {\ldots}       {\ldots}
329  0.312270  0.918388  0.377515  0.631530  0.823016
330  0.264547  0.479814  0.647478  0.813936  0.374694
331  0.267763  0.443711  0.342104  0.796020  0.879930
332  0.929596  0.795114  0.602521  0.842777  0.880249
333  0.221595  0.760004  0.909669  0.105035  0.689167

[334 rows x 5 columns]
\end{Verbatim}
\end{tcolorbox}
        
    \begin{tcolorbox}[breakable, size=fbox, boxrule=1pt, pad at break*=1mm,colback=cellbackground, colframe=cellborder]
\prompt{In}{incolor}{23}{\boxspacing}
\begin{Verbatim}[commandchars=\\\{\}]
\PY{n+nb}{type}\PY{p}{(}\PY{n}{newdfff}\PY{p}{)} \PY{c+c1}{\PYZsh{} bata raha ki ye dataframe hai \PYZsh{}}
\end{Verbatim}
\end{tcolorbox}

            \begin{tcolorbox}[breakable, size=fbox, boxrule=.5pt, pad at break*=1mm, opacityfill=0]
\prompt{Out}{outcolor}{23}{\boxspacing}
\begin{Verbatim}[commandchars=\\\{\}]
pandas.core.frame.DataFrame
\end{Verbatim}
\end{tcolorbox}
        
    \begin{tcolorbox}[breakable, size=fbox, boxrule=1pt, pad at break*=1mm,colback=cellbackground, colframe=cellborder]
\prompt{In}{incolor}{24}{\boxspacing}
\begin{Verbatim}[commandchars=\\\{\}]
\PY{n}{newdfff}\PY{o}{.}\PY{n}{describe}\PY{p}{(}\PY{p}{)}
\end{Verbatim}
\end{tcolorbox}

            \begin{tcolorbox}[breakable, size=fbox, boxrule=.5pt, pad at break*=1mm, opacityfill=0]
\prompt{Out}{outcolor}{24}{\boxspacing}
\begin{Verbatim}[commandchars=\\\{\}]
                0           1           2           3           4
count  334.000000  334.000000  334.000000  334.000000  334.000000
mean     0.492469    0.476737    0.507156    0.482960    0.526656
std      0.284950    0.290641    0.289953    0.284042    0.273981
min      0.000334    0.000021    0.002552    0.007948    0.002052
25\%      0.246542    0.216117    0.260031    0.237201    0.321162
50\%      0.497922    0.459141    0.522007    0.492353    0.536820
75\%      0.730725    0.744300    0.761693    0.708372    0.734478
max      0.999539    0.998868    0.998139    0.997852    0.998499
\end{Verbatim}
\end{tcolorbox}
        
    \begin{tcolorbox}[breakable, size=fbox, boxrule=1pt, pad at break*=1mm,colback=cellbackground, colframe=cellborder]
\prompt{In}{incolor}{25}{\boxspacing}
\begin{Verbatim}[commandchars=\\\{\}]
\PY{n}{newdfff}\PY{o}{.}\PY{n}{dtypes}
\end{Verbatim}
\end{tcolorbox}

            \begin{tcolorbox}[breakable, size=fbox, boxrule=.5pt, pad at break*=1mm, opacityfill=0]
\prompt{Out}{outcolor}{25}{\boxspacing}
\begin{Verbatim}[commandchars=\\\{\}]
0    float64
1    float64
2    float64
3    float64
4    float64
dtype: object
\end{Verbatim}
\end{tcolorbox}
        
    \begin{tcolorbox}[breakable, size=fbox, boxrule=1pt, pad at break*=1mm,colback=cellbackground, colframe=cellborder]
\prompt{In}{incolor}{26}{\boxspacing}
\begin{Verbatim}[commandchars=\\\{\}]
\PY{n}{newdfff}\PY{o}{.}\PY{n}{head}\PY{p}{(}\PY{l+m+mi}{10}\PY{p}{)}
\end{Verbatim}
\end{tcolorbox}

            \begin{tcolorbox}[breakable, size=fbox, boxrule=.5pt, pad at break*=1mm, opacityfill=0]
\prompt{Out}{outcolor}{26}{\boxspacing}
\begin{Verbatim}[commandchars=\\\{\}]
          0         1         2         3         4
0  0.781981  0.160657  0.306927  0.597240  0.817418
1  0.337682  0.032490  0.578410  0.196446  0.459194
2  0.999539  0.553314  0.687496  0.716025  0.397865
3  0.092922  0.507143  0.230202  0.512950  0.298372
4  0.658756  0.129097  0.090615  0.637531  0.647706
5  0.245853  0.725398  0.840809  0.026238  0.895827
6  0.711601  0.520287  0.547394  0.698721  0.618776
7  0.754726  0.745486  0.389523  0.512943  0.874342
8  0.899062  0.542402  0.582715  0.965121  0.500594
9  0.699753  0.239546  0.955863  0.484494  0.592837
\end{Verbatim}
\end{tcolorbox}
        
    \begin{tcolorbox}[breakable, size=fbox, boxrule=1pt, pad at break*=1mm,colback=cellbackground, colframe=cellborder]
\prompt{In}{incolor}{27}{\boxspacing}
\begin{Verbatim}[commandchars=\\\{\}]
\PY{n}{newdfff}\PY{p}{[}\PY{l+m+mi}{0}\PY{p}{]}\PY{p}{[}\PY{l+m+mi}{0}\PY{p}{]}\PY{o}{=}\PY{l+s+s1}{\PYZsq{}}\PY{l+s+s1}{shivam patel}\PY{l+s+s1}{\PYZsq{}}
\end{Verbatim}
\end{tcolorbox}

    \begin{tcolorbox}[breakable, size=fbox, boxrule=1pt, pad at break*=1mm,colback=cellbackground, colframe=cellborder]
\prompt{In}{incolor}{28}{\boxspacing}
\begin{Verbatim}[commandchars=\\\{\}]
\PY{n}{newdfff}
\end{Verbatim}
\end{tcolorbox}

            \begin{tcolorbox}[breakable, size=fbox, boxrule=.5pt, pad at break*=1mm, opacityfill=0]
\prompt{Out}{outcolor}{28}{\boxspacing}
\begin{Verbatim}[commandchars=\\\{\}]
                0         1         2         3         4
0    shivam patel  0.160657  0.306927  0.597240  0.817418
1        0.337682  0.032490  0.578410  0.196446  0.459194
2        0.999539  0.553314  0.687496  0.716025  0.397865
3       0.0929223  0.507143  0.230202  0.512950  0.298372
4        0.658756  0.129097  0.090615  0.637531  0.647706
..            {\ldots}       {\ldots}       {\ldots}       {\ldots}       {\ldots}
329       0.31227  0.918388  0.377515  0.631530  0.823016
330      0.264547  0.479814  0.647478  0.813936  0.374694
331      0.267763  0.443711  0.342104  0.796020  0.879930
332      0.929596  0.795114  0.602521  0.842777  0.880249
333      0.221595  0.760004  0.909669  0.105035  0.689167

[334 rows x 5 columns]
\end{Verbatim}
\end{tcolorbox}
        
    \begin{tcolorbox}[breakable, size=fbox, boxrule=1pt, pad at break*=1mm,colback=cellbackground, colframe=cellborder]
\prompt{In}{incolor}{29}{\boxspacing}
\begin{Verbatim}[commandchars=\\\{\}]
\PY{n}{newdfff}\PY{o}{.}\PY{n}{dtypes} \PY{c+c1}{\PYZsh{} first row data types changes to object\PYZsh{}\PYZsh{}}
\end{Verbatim}
\end{tcolorbox}

            \begin{tcolorbox}[breakable, size=fbox, boxrule=.5pt, pad at break*=1mm, opacityfill=0]
\prompt{Out}{outcolor}{29}{\boxspacing}
\begin{Verbatim}[commandchars=\\\{\}]
0     object
1    float64
2    float64
3    float64
4    float64
dtype: object
\end{Verbatim}
\end{tcolorbox}
        
    \begin{tcolorbox}[breakable, size=fbox, boxrule=1pt, pad at break*=1mm,colback=cellbackground, colframe=cellborder]
\prompt{In}{incolor}{30}{\boxspacing}
\begin{Verbatim}[commandchars=\\\{\}]
\PY{n}{newdfff}\PY{o}{.}\PY{n}{index}
\end{Verbatim}
\end{tcolorbox}

            \begin{tcolorbox}[breakable, size=fbox, boxrule=.5pt, pad at break*=1mm, opacityfill=0]
\prompt{Out}{outcolor}{30}{\boxspacing}
\begin{Verbatim}[commandchars=\\\{\}]
Int64Index([  0,   1,   2,   3,   4,   5,   6,   7,   8,   9,
            {\ldots}
            324, 325, 326, 327, 328, 329, 330, 331, 332, 333],
           dtype='int64', length=334)
\end{Verbatim}
\end{tcolorbox}
        
    \begin{tcolorbox}[breakable, size=fbox, boxrule=1pt, pad at break*=1mm,colback=cellbackground, colframe=cellborder]
\prompt{In}{incolor}{31}{\boxspacing}
\begin{Verbatim}[commandchars=\\\{\}]
\PY{n}{newdfff}\PY{o}{.}\PY{n}{columns}
\end{Verbatim}
\end{tcolorbox}

            \begin{tcolorbox}[breakable, size=fbox, boxrule=.5pt, pad at break*=1mm, opacityfill=0]
\prompt{Out}{outcolor}{31}{\boxspacing}
\begin{Verbatim}[commandchars=\\\{\}]
RangeIndex(start=0, stop=5, step=1)
\end{Verbatim}
\end{tcolorbox}
        
    \begin{tcolorbox}[breakable, size=fbox, boxrule=1pt, pad at break*=1mm,colback=cellbackground, colframe=cellborder]
\prompt{In}{incolor}{32}{\boxspacing}
\begin{Verbatim}[commandchars=\\\{\}]
\PY{n}{newdfff}\PY{o}{.}\PY{n}{to\PYZus{}numpy}\PY{p}{(}\PY{p}{)} \PY{c+c1}{\PYZsh{} datatypes converting in to numpy arry\PYZsh{}}
\end{Verbatim}
\end{tcolorbox}

            \begin{tcolorbox}[breakable, size=fbox, boxrule=.5pt, pad at break*=1mm, opacityfill=0]
\prompt{Out}{outcolor}{32}{\boxspacing}
\begin{Verbatim}[commandchars=\\\{\}]
array([['shivam patel', 0.16065673709301487, 0.30692659284693935,
        0.5972397973500868, 0.8174182312622135],
       [0.33768207911931425, 0.03248971190293992, 0.5784102906470643,
        0.19644629680681125, 0.45919430561115704],
       [0.9995392699434426, 0.5533142668440891, 0.687495615125021,
        0.7160247552726826, 0.39786514231437353],
       {\ldots},
       [0.26776257289656713, 0.44371140024431166, 0.34210377351550014,
        0.7960200907114762, 0.879930068879511],
       [0.9295957702894861, 0.7951142178287649, 0.602520754163023,
        0.8427772992785751, 0.8802486322948865],
       [0.22159515130889895, 0.7600039998877186, 0.9096688636965189,
        0.10503494605817776, 0.6891668779946094]], dtype=object)
\end{Verbatim}
\end{tcolorbox}
        
    \begin{tcolorbox}[breakable, size=fbox, boxrule=1pt, pad at break*=1mm,colback=cellbackground, colframe=cellborder]
\prompt{In}{incolor}{33}{\boxspacing}
\begin{Verbatim}[commandchars=\\\{\}]
\PY{n}{newdfff}\PY{o}{.}\PY{n}{T} \PY{c+c1}{\PYZsh{} transpose of matrix\PYZsh{}}
\end{Verbatim}
\end{tcolorbox}

            \begin{tcolorbox}[breakable, size=fbox, boxrule=.5pt, pad at break*=1mm, opacityfill=0]
\prompt{Out}{outcolor}{33}{\boxspacing}
\begin{Verbatim}[commandchars=\\\{\}]
            0          1         2          3          4          5    \textbackslash{}
0  shivam patel   0.337682  0.999539  0.0929223   0.658756   0.245853
1      0.160657  0.0324897  0.553314   0.507143   0.129097   0.725398
2      0.306927    0.57841  0.687496   0.230202  0.0906149   0.840809
3       0.59724   0.196446  0.716025    0.51295   0.637531  0.0262376
4      0.817418   0.459194  0.397865   0.298372   0.647706   0.895827

        6         7         8         9    {\ldots}        324        325  \textbackslash{}
0  0.711601  0.754726  0.899062  0.699753  {\ldots}  0.0448656   0.354321
1  0.520287  0.745486  0.542402  0.239546  {\ldots}    0.56699   0.353269
2  0.547394  0.389523  0.582715  0.955863  {\ldots}  0.0709526   0.444389
3  0.698721  0.512943  0.965121  0.484494  {\ldots}  0.0187626  0.0943753
4  0.618776  0.874342  0.500594  0.592837  {\ldots}    0.47963   0.694159

        326        327        328       329       330       331       332  \textbackslash{}
0  0.688913   0.320678  0.0959358   0.31227  0.264547  0.267763  0.929596
1  0.633718   0.478718   0.136939  0.918388  0.479814  0.443711  0.795114
2  0.613868   0.846737   0.122672  0.377515  0.647478  0.342104  0.602521
3  0.506235   0.397915   0.114118   0.63153  0.813936   0.79602  0.842777
4  0.344469  0.0398603   0.911283  0.823016  0.374694   0.87993  0.880249

        333
0  0.221595
1  0.760004
2  0.909669
3  0.105035
4  0.689167

[5 rows x 334 columns]
\end{Verbatim}
\end{tcolorbox}
        
    \begin{tcolorbox}[breakable, size=fbox, boxrule=1pt, pad at break*=1mm,colback=cellbackground, colframe=cellborder]
\prompt{In}{incolor}{34}{\boxspacing}
\begin{Verbatim}[commandchars=\\\{\}]
\PY{n}{newdfff}\PY{o}{.}\PY{n}{head}\PY{p}{(}\PY{p}{)}
\end{Verbatim}
\end{tcolorbox}

            \begin{tcolorbox}[breakable, size=fbox, boxrule=.5pt, pad at break*=1mm, opacityfill=0]
\prompt{Out}{outcolor}{34}{\boxspacing}
\begin{Verbatim}[commandchars=\\\{\}]
              0         1         2         3         4
0  shivam patel  0.160657  0.306927  0.597240  0.817418
1      0.337682  0.032490  0.578410  0.196446  0.459194
2      0.999539  0.553314  0.687496  0.716025  0.397865
3     0.0929223  0.507143  0.230202  0.512950  0.298372
4      0.658756  0.129097  0.090615  0.637531  0.647706
\end{Verbatim}
\end{tcolorbox}
        
    \begin{tcolorbox}[breakable, size=fbox, boxrule=1pt, pad at break*=1mm,colback=cellbackground, colframe=cellborder]
\prompt{In}{incolor}{35}{\boxspacing}
\begin{Verbatim}[commandchars=\\\{\}]
\PY{n}{newdfff}\PY{o}{.}\PY{n}{sort\PYZus{}index}\PY{p}{(}\PY{n}{axis}\PY{o}{=}\PY{l+m+mi}{1}\PY{p}{,}\PY{n}{ascending}\PY{o}{=} \PY{k+kc}{False}\PY{p}{)} \PY{c+c1}{\PYZsh{} axis0 means row and axis 1 means coloum , and here we can using sort\PYZus{}index function\PYZsh{}}
\end{Verbatim}
\end{tcolorbox}

            \begin{tcolorbox}[breakable, size=fbox, boxrule=.5pt, pad at break*=1mm, opacityfill=0]
\prompt{Out}{outcolor}{35}{\boxspacing}
\begin{Verbatim}[commandchars=\\\{\}]
            4         3         2         1             0
0    0.817418  0.597240  0.306927  0.160657  shivam patel
1    0.459194  0.196446  0.578410  0.032490      0.337682
2    0.397865  0.716025  0.687496  0.553314      0.999539
3    0.298372  0.512950  0.230202  0.507143     0.0929223
4    0.647706  0.637531  0.090615  0.129097      0.658756
..        {\ldots}       {\ldots}       {\ldots}       {\ldots}           {\ldots}
329  0.823016  0.631530  0.377515  0.918388       0.31227
330  0.374694  0.813936  0.647478  0.479814      0.264547
331  0.879930  0.796020  0.342104  0.443711      0.267763
332  0.880249  0.842777  0.602521  0.795114      0.929596
333  0.689167  0.105035  0.909669  0.760004      0.221595

[334 rows x 5 columns]
\end{Verbatim}
\end{tcolorbox}
        
    \begin{tcolorbox}[breakable, size=fbox, boxrule=1pt, pad at break*=1mm,colback=cellbackground, colframe=cellborder]
\prompt{In}{incolor}{36}{\boxspacing}
\begin{Verbatim}[commandchars=\\\{\}]
\PY{n}{newdfff}\PY{o}{.}\PY{n}{sort\PYZus{}index}\PY{p}{(}\PY{n}{axis}\PY{o}{=}\PY{l+m+mi}{0}\PY{p}{,}\PY{n}{ascending}\PY{o}{=} \PY{k+kc}{False}\PY{p}{)} \PY{c+c1}{\PYZsh{} sort from 333 to 0\PYZsh{}}
\end{Verbatim}
\end{tcolorbox}

            \begin{tcolorbox}[breakable, size=fbox, boxrule=.5pt, pad at break*=1mm, opacityfill=0]
\prompt{Out}{outcolor}{36}{\boxspacing}
\begin{Verbatim}[commandchars=\\\{\}]
                0         1         2         3         4
333      0.221595  0.760004  0.909669  0.105035  0.689167
332      0.929596  0.795114  0.602521  0.842777  0.880249
331      0.267763  0.443711  0.342104  0.796020  0.879930
330      0.264547  0.479814  0.647478  0.813936  0.374694
329       0.31227  0.918388  0.377515  0.631530  0.823016
..            {\ldots}       {\ldots}       {\ldots}       {\ldots}       {\ldots}
4        0.658756  0.129097  0.090615  0.637531  0.647706
3       0.0929223  0.507143  0.230202  0.512950  0.298372
2        0.999539  0.553314  0.687496  0.716025  0.397865
1        0.337682  0.032490  0.578410  0.196446  0.459194
0    shivam patel  0.160657  0.306927  0.597240  0.817418

[334 rows x 5 columns]
\end{Verbatim}
\end{tcolorbox}
        
    \begin{tcolorbox}[breakable, size=fbox, boxrule=1pt, pad at break*=1mm,colback=cellbackground, colframe=cellborder]
\prompt{In}{incolor}{37}{\boxspacing}
\begin{Verbatim}[commandchars=\\\{\}]
\PY{n+nb}{type}\PY{p}{(}\PY{n}{newdfff}\PY{p}{[}\PY{l+m+mi}{0}\PY{p}{]}\PY{p}{)}
\end{Verbatim}
\end{tcolorbox}

            \begin{tcolorbox}[breakable, size=fbox, boxrule=.5pt, pad at break*=1mm, opacityfill=0]
\prompt{Out}{outcolor}{37}{\boxspacing}
\begin{Verbatim}[commandchars=\\\{\}]
pandas.core.series.Series
\end{Verbatim}
\end{tcolorbox}
        
    \begin{tcolorbox}[breakable, size=fbox, boxrule=1pt, pad at break*=1mm,colback=cellbackground, colframe=cellborder]
\prompt{In}{incolor}{38}{\boxspacing}
\begin{Verbatim}[commandchars=\\\{\}]
\PY{n}{newdfff}
\end{Verbatim}
\end{tcolorbox}

            \begin{tcolorbox}[breakable, size=fbox, boxrule=.5pt, pad at break*=1mm, opacityfill=0]
\prompt{Out}{outcolor}{38}{\boxspacing}
\begin{Verbatim}[commandchars=\\\{\}]
                0         1         2         3         4
0    shivam patel  0.160657  0.306927  0.597240  0.817418
1        0.337682  0.032490  0.578410  0.196446  0.459194
2        0.999539  0.553314  0.687496  0.716025  0.397865
3       0.0929223  0.507143  0.230202  0.512950  0.298372
4        0.658756  0.129097  0.090615  0.637531  0.647706
..            {\ldots}       {\ldots}       {\ldots}       {\ldots}       {\ldots}
329       0.31227  0.918388  0.377515  0.631530  0.823016
330      0.264547  0.479814  0.647478  0.813936  0.374694
331      0.267763  0.443711  0.342104  0.796020  0.879930
332      0.929596  0.795114  0.602521  0.842777  0.880249
333      0.221595  0.760004  0.909669  0.105035  0.689167

[334 rows x 5 columns]
\end{Verbatim}
\end{tcolorbox}
        
    \begin{tcolorbox}[breakable, size=fbox, boxrule=1pt, pad at break*=1mm,colback=cellbackground, colframe=cellborder]
\prompt{In}{incolor}{39}{\boxspacing}
\begin{Verbatim}[commandchars=\\\{\}]
\PY{n}{newdf} \PY{o}{=} \PY{n}{newdfff}
\end{Verbatim}
\end{tcolorbox}

    \begin{tcolorbox}[breakable, size=fbox, boxrule=1pt, pad at break*=1mm,colback=cellbackground, colframe=cellborder]
\prompt{In}{incolor}{40}{\boxspacing}
\begin{Verbatim}[commandchars=\\\{\}]
\PY{n}{newdf}
\end{Verbatim}
\end{tcolorbox}

            \begin{tcolorbox}[breakable, size=fbox, boxrule=.5pt, pad at break*=1mm, opacityfill=0]
\prompt{Out}{outcolor}{40}{\boxspacing}
\begin{Verbatim}[commandchars=\\\{\}]
                0         1         2         3         4
0    shivam patel  0.160657  0.306927  0.597240  0.817418
1        0.337682  0.032490  0.578410  0.196446  0.459194
2        0.999539  0.553314  0.687496  0.716025  0.397865
3       0.0929223  0.507143  0.230202  0.512950  0.298372
4        0.658756  0.129097  0.090615  0.637531  0.647706
..            {\ldots}       {\ldots}       {\ldots}       {\ldots}       {\ldots}
329       0.31227  0.918388  0.377515  0.631530  0.823016
330      0.264547  0.479814  0.647478  0.813936  0.374694
331      0.267763  0.443711  0.342104  0.796020  0.879930
332      0.929596  0.795114  0.602521  0.842777  0.880249
333      0.221595  0.760004  0.909669  0.105035  0.689167

[334 rows x 5 columns]
\end{Verbatim}
\end{tcolorbox}
        
    \begin{tcolorbox}[breakable, size=fbox, boxrule=1pt, pad at break*=1mm,colback=cellbackground, colframe=cellborder]
\prompt{In}{incolor}{41}{\boxspacing}
\begin{Verbatim}[commandchars=\\\{\}]
\PY{n}{newdf}\PY{o}{.}\PY{n}{loc}\PY{p}{[}\PY{l+m+mi}{0}\PY{p}{,}\PY{l+m+mi}{1}\PY{p}{]} \PY{o}{=} \PY{l+m+mi}{4}
\end{Verbatim}
\end{tcolorbox}

    \begin{tcolorbox}[breakable, size=fbox, boxrule=1pt, pad at break*=1mm,colback=cellbackground, colframe=cellborder]
\prompt{In}{incolor}{42}{\boxspacing}
\begin{Verbatim}[commandchars=\\\{\}]
\PY{n}{newdf}
\end{Verbatim}
\end{tcolorbox}

            \begin{tcolorbox}[breakable, size=fbox, boxrule=.5pt, pad at break*=1mm, opacityfill=0]
\prompt{Out}{outcolor}{42}{\boxspacing}
\begin{Verbatim}[commandchars=\\\{\}]
                0         1         2         3         4
0    shivam patel  4.000000  0.306927  0.597240  0.817418
1        0.337682  0.032490  0.578410  0.196446  0.459194
2        0.999539  0.553314  0.687496  0.716025  0.397865
3       0.0929223  0.507143  0.230202  0.512950  0.298372
4        0.658756  0.129097  0.090615  0.637531  0.647706
..            {\ldots}       {\ldots}       {\ldots}       {\ldots}       {\ldots}
329       0.31227  0.918388  0.377515  0.631530  0.823016
330      0.264547  0.479814  0.647478  0.813936  0.374694
331      0.267763  0.443711  0.342104  0.796020  0.879930
332      0.929596  0.795114  0.602521  0.842777  0.880249
333      0.221595  0.760004  0.909669  0.105035  0.689167

[334 rows x 5 columns]
\end{Verbatim}
\end{tcolorbox}
        
    \begin{tcolorbox}[breakable, size=fbox, boxrule=1pt, pad at break*=1mm,colback=cellbackground, colframe=cellborder]
\prompt{In}{incolor}{46}{\boxspacing}
\begin{Verbatim}[commandchars=\\\{\}]
\PY{n}{newdf}\PY{o}{.}\PY{n}{loc}\PY{p}{[}\PY{l+m+mi}{3}\PY{p}{,}\PY{l+m+mi}{3}\PY{p}{]} \PY{o}{=} \PY{l+m+mi}{8}
\end{Verbatim}
\end{tcolorbox}

    \begin{tcolorbox}[breakable, size=fbox, boxrule=1pt, pad at break*=1mm,colback=cellbackground, colframe=cellborder]
\prompt{In}{incolor}{47}{\boxspacing}
\begin{Verbatim}[commandchars=\\\{\}]
\PY{n}{newdf}
\end{Verbatim}
\end{tcolorbox}

            \begin{tcolorbox}[breakable, size=fbox, boxrule=.5pt, pad at break*=1mm, opacityfill=0]
\prompt{Out}{outcolor}{47}{\boxspacing}
\begin{Verbatim}[commandchars=\\\{\}]
                0         1         2         3         4
0    shivam patel  4.000000  0.306927  8.000000  0.817418
1        0.337682  0.032490  0.578410  0.196446  0.459194
2        0.999539  0.553314  0.687496  0.716025  0.397865
3       0.0929223  0.507143  0.230202  8.000000  0.298372
4        0.658756  0.129097  0.090615  0.637531  0.647706
..            {\ldots}       {\ldots}       {\ldots}       {\ldots}       {\ldots}
329       0.31227  0.918388  0.377515  0.631530  0.823016
330      0.264547  0.479814  0.647478  0.813936  0.374694
331      0.267763  0.443711  0.342104  0.796020  0.879930
332      0.929596  0.795114  0.602521  0.842777  0.880249
333      0.221595  0.760004  0.909669  0.105035  0.689167

[334 rows x 5 columns]
\end{Verbatim}
\end{tcolorbox}
        
    \begin{tcolorbox}[breakable, size=fbox, boxrule=1pt, pad at break*=1mm,colback=cellbackground, colframe=cellborder]
\prompt{In}{incolor}{48}{\boxspacing}
\begin{Verbatim}[commandchars=\\\{\}]
\PY{n}{newdf}\PY{o}{.}\PY{n}{columns} \PY{o}{=} \PY{n+nb}{list}\PY{p}{(}\PY{l+s+s1}{\PYZsq{}}\PY{l+s+s1}{abcde}\PY{l+s+s1}{\PYZsq{}}\PY{p}{)}
\end{Verbatim}
\end{tcolorbox}

    \begin{tcolorbox}[breakable, size=fbox, boxrule=1pt, pad at break*=1mm,colback=cellbackground, colframe=cellborder]
\prompt{In}{incolor}{49}{\boxspacing}
\begin{Verbatim}[commandchars=\\\{\}]
\PY{n}{newdf}
\end{Verbatim}
\end{tcolorbox}

            \begin{tcolorbox}[breakable, size=fbox, boxrule=.5pt, pad at break*=1mm, opacityfill=0]
\prompt{Out}{outcolor}{49}{\boxspacing}
\begin{Verbatim}[commandchars=\\\{\}]
                a         b         c         d         e
0    shivam patel  4.000000  0.306927  8.000000  0.817418
1        0.337682  0.032490  0.578410  0.196446  0.459194
2        0.999539  0.553314  0.687496  0.716025  0.397865
3       0.0929223  0.507143  0.230202  8.000000  0.298372
4        0.658756  0.129097  0.090615  0.637531  0.647706
..            {\ldots}       {\ldots}       {\ldots}       {\ldots}       {\ldots}
329       0.31227  0.918388  0.377515  0.631530  0.823016
330      0.264547  0.479814  0.647478  0.813936  0.374694
331      0.267763  0.443711  0.342104  0.796020  0.879930
332      0.929596  0.795114  0.602521  0.842777  0.880249
333      0.221595  0.760004  0.909669  0.105035  0.689167

[334 rows x 5 columns]
\end{Verbatim}
\end{tcolorbox}
        
    \begin{tcolorbox}[breakable, size=fbox, boxrule=1pt, pad at break*=1mm,colback=cellbackground, colframe=cellborder]
\prompt{In}{incolor}{50}{\boxspacing}
\begin{Verbatim}[commandchars=\\\{\}]
\PY{n}{newdf}\PY{o}{.}\PY{n}{loc}\PY{p}{[}\PY{l+m+mi}{0}\PY{p}{,}\PY{l+m+mi}{0}\PY{p}{]} \PY{o}{=} \PY{l+m+mi}{78}
\end{Verbatim}
\end{tcolorbox}

    \begin{tcolorbox}[breakable, size=fbox, boxrule=1pt, pad at break*=1mm,colback=cellbackground, colframe=cellborder]
\prompt{In}{incolor}{51}{\boxspacing}
\begin{Verbatim}[commandchars=\\\{\}]
\PY{n}{newdf}
\end{Verbatim}
\end{tcolorbox}

            \begin{tcolorbox}[breakable, size=fbox, boxrule=.5pt, pad at break*=1mm, opacityfill=0]
\prompt{Out}{outcolor}{51}{\boxspacing}
\begin{Verbatim}[commandchars=\\\{\}]
                a         b         c         d         e     0
0    shivam patel  4.000000  0.306927  8.000000  0.817418  78.0
1        0.337682  0.032490  0.578410  0.196446  0.459194   NaN
2        0.999539  0.553314  0.687496  0.716025  0.397865   NaN
3       0.0929223  0.507143  0.230202  8.000000  0.298372   NaN
4        0.658756  0.129097  0.090615  0.637531  0.647706   NaN
..            {\ldots}       {\ldots}       {\ldots}       {\ldots}       {\ldots}   {\ldots}
329       0.31227  0.918388  0.377515  0.631530  0.823016   NaN
330      0.264547  0.479814  0.647478  0.813936  0.374694   NaN
331      0.267763  0.443711  0.342104  0.796020  0.879930   NaN
332      0.929596  0.795114  0.602521  0.842777  0.880249   NaN
333      0.221595  0.760004  0.909669  0.105035  0.689167   NaN

[334 rows x 6 columns]
\end{Verbatim}
\end{tcolorbox}
        
    \begin{tcolorbox}[breakable, size=fbox, boxrule=1pt, pad at break*=1mm,colback=cellbackground, colframe=cellborder]
\prompt{In}{incolor}{58}{\boxspacing}
\begin{Verbatim}[commandchars=\\\{\}]
\PY{n}{newdf} \PY{o}{=} \PY{n}{newdf}\PY{o}{.}\PY{n}{drop}\PY{p}{(}\PY{l+m+mi}{0}\PY{p}{,}\PY{n}{axis}\PY{o}{=}\PY{l+m+mi}{1}\PY{p}{)}
\end{Verbatim}
\end{tcolorbox}

    \begin{tcolorbox}[breakable, size=fbox, boxrule=1pt, pad at break*=1mm,colback=cellbackground, colframe=cellborder]
\prompt{In}{incolor}{59}{\boxspacing}
\begin{Verbatim}[commandchars=\\\{\}]
\PY{n}{newdf}
\end{Verbatim}
\end{tcolorbox}

            \begin{tcolorbox}[breakable, size=fbox, boxrule=.5pt, pad at break*=1mm, opacityfill=0]
\prompt{Out}{outcolor}{59}{\boxspacing}
\begin{Verbatim}[commandchars=\\\{\}]
                a         b         c         d         e
0    shivam patel  4.000000  0.306927  8.000000  0.817418
1        0.337682  0.032490  0.578410  0.196446  0.459194
2        0.999539  0.553314  0.687496  0.716025  0.397865
3       0.0929223  0.507143  0.230202  8.000000  0.298372
4        0.658756  0.129097  0.090615  0.637531  0.647706
..            {\ldots}       {\ldots}       {\ldots}       {\ldots}       {\ldots}
329       0.31227  0.918388  0.377515  0.631530  0.823016
330      0.264547  0.479814  0.647478  0.813936  0.374694
331      0.267763  0.443711  0.342104  0.796020  0.879930
332      0.929596  0.795114  0.602521  0.842777  0.880249
333      0.221595  0.760004  0.909669  0.105035  0.689167

[334 rows x 5 columns]
\end{Verbatim}
\end{tcolorbox}
        
    \begin{tcolorbox}[breakable, size=fbox, boxrule=1pt, pad at break*=1mm,colback=cellbackground, colframe=cellborder]
\prompt{In}{incolor}{61}{\boxspacing}
\begin{Verbatim}[commandchars=\\\{\}]
\PY{n}{newdf}\PY{o}{.}\PY{n}{iloc}\PY{p}{[}\PY{l+m+mi}{0}\PY{p}{,}\PY{l+m+mi}{4}\PY{p}{]}
\end{Verbatim}
\end{tcolorbox}

            \begin{tcolorbox}[breakable, size=fbox, boxrule=.5pt, pad at break*=1mm, opacityfill=0]
\prompt{Out}{outcolor}{61}{\boxspacing}
\begin{Verbatim}[commandchars=\\\{\}]
0.8174182312622135
\end{Verbatim}
\end{tcolorbox}
        
    \begin{tcolorbox}[breakable, size=fbox, boxrule=1pt, pad at break*=1mm,colback=cellbackground, colframe=cellborder]
\prompt{In}{incolor}{64}{\boxspacing}
\begin{Verbatim}[commandchars=\\\{\}]
\PY{n}{newdf}\PY{o}{.}\PY{n}{iloc}\PY{p}{[}\PY{p}{[}\PY{l+m+mi}{0}\PY{p}{,}\PY{l+m+mi}{1}\PY{p}{,}\PY{l+m+mi}{4}\PY{p}{,}\PY{l+m+mi}{5}\PY{p}{,}\PY{l+m+mi}{55}\PY{p}{,}\PY{l+m+mi}{48}\PY{p}{,}\PY{l+m+mi}{234}\PY{p}{]}\PY{p}{,}\PY{p}{[}\PY{l+m+mi}{2}\PY{p}{,}\PY{l+m+mi}{4}\PY{p}{]}\PY{p}{]}
\end{Verbatim}
\end{tcolorbox}

            \begin{tcolorbox}[breakable, size=fbox, boxrule=.5pt, pad at break*=1mm, opacityfill=0]
\prompt{Out}{outcolor}{64}{\boxspacing}
\begin{Verbatim}[commandchars=\\\{\}]
            c         e
0    0.306927  0.817418
1    0.578410  0.459194
4    0.090615  0.647706
5    0.840809  0.895827
55   0.793632  0.662192
48   0.249413  0.474844
234  0.199555  0.588891
\end{Verbatim}
\end{tcolorbox}
        
    \begin{tcolorbox}[breakable, size=fbox, boxrule=1pt, pad at break*=1mm,colback=cellbackground, colframe=cellborder]
\prompt{In}{incolor}{67}{\boxspacing}
\begin{Verbatim}[commandchars=\\\{\}]
\PY{n}{newdf}\PY{o}{.}\PY{n}{loc}\PY{p}{[}\PY{p}{[}\PY{l+m+mi}{1}\PY{p}{,}\PY{l+m+mi}{2}\PY{p}{,}\PY{l+m+mi}{3}\PY{p}{]}\PY{p}{,}\PY{p}{[}\PY{l+s+s1}{\PYZsq{}}\PY{l+s+s1}{a}\PY{l+s+s1}{\PYZsq{}}\PY{p}{,}\PY{l+s+s1}{\PYZsq{}}\PY{l+s+s1}{c}\PY{l+s+s1}{\PYZsq{}}\PY{p}{]}\PY{p}{]}\PY{c+c1}{\PYZsh{}selecting perticuler row and coloum\PYZsh{}}
\end{Verbatim}
\end{tcolorbox}

            \begin{tcolorbox}[breakable, size=fbox, boxrule=.5pt, pad at break*=1mm, opacityfill=0]
\prompt{Out}{outcolor}{67}{\boxspacing}
\begin{Verbatim}[commandchars=\\\{\}]
           a         c
1   0.337682  0.578410
2   0.999539  0.687496
3  0.0929223  0.230202
\end{Verbatim}
\end{tcolorbox}
        
    \begin{tcolorbox}[breakable, size=fbox, boxrule=1pt, pad at break*=1mm,colback=cellbackground, colframe=cellborder]
\prompt{In}{incolor}{68}{\boxspacing}
\begin{Verbatim}[commandchars=\\\{\}]
\PY{n}{newdf}
\end{Verbatim}
\end{tcolorbox}

            \begin{tcolorbox}[breakable, size=fbox, boxrule=.5pt, pad at break*=1mm, opacityfill=0]
\prompt{Out}{outcolor}{68}{\boxspacing}
\begin{Verbatim}[commandchars=\\\{\}]
                a         b         c         d         e
0    shivam patel  4.000000  0.306927  8.000000  0.817418
1        0.337682  0.032490  0.578410  0.196446  0.459194
2        0.999539  0.553314  0.687496  0.716025  0.397865
3       0.0929223  0.507143  0.230202  8.000000  0.298372
4        0.658756  0.129097  0.090615  0.637531  0.647706
..            {\ldots}       {\ldots}       {\ldots}       {\ldots}       {\ldots}
329       0.31227  0.918388  0.377515  0.631530  0.823016
330      0.264547  0.479814  0.647478  0.813936  0.374694
331      0.267763  0.443711  0.342104  0.796020  0.879930
332      0.929596  0.795114  0.602521  0.842777  0.880249
333      0.221595  0.760004  0.909669  0.105035  0.689167

[334 rows x 5 columns]
\end{Verbatim}
\end{tcolorbox}
        
    \begin{tcolorbox}[breakable, size=fbox, boxrule=1pt, pad at break*=1mm,colback=cellbackground, colframe=cellborder]
\prompt{In}{incolor}{80}{\boxspacing}
\begin{Verbatim}[commandchars=\\\{\}]
\PY{n}{newdf}\PY{o}{.}\PY{n}{loc}\PY{p}{[}\PY{p}{(}\PY{n}{newdf}\PY{p}{[}\PY{l+s+s1}{\PYZsq{}}\PY{l+s+s1}{a}\PY{l+s+s1}{\PYZsq{}}\PY{p}{]}\PY{o}{\PYZlt{}}\PY{l+m+mf}{0.3}\PY{p}{)} \PY{o+ow}{and} \PY{p}{(}\PY{n}{newdf}\PY{p}{[}\PY{l+s+s1}{\PYZsq{}}\PY{l+s+s1}{b}\PY{l+s+s1}{\PYZsq{}}\PY{p}{]}\PY{o}{\PYZgt{}}\PY{l+m+mf}{0.2}\PY{p}{)}\PY{p}{]}\PY{c+c1}{\PYZsh{}\PYZsh{}this function is use to give condition \PYZsh{}\PYZsh{}}
\end{Verbatim}
\end{tcolorbox}

    \begin{Verbatim}[commandchars=\\\{\}, frame=single, framerule=2mm, rulecolor=\color{outerrorbackground}]
\textcolor{ansi-red-intense}{\textbf{---------------------------------------------------------------------------}}
\textcolor{ansi-red-intense}{\textbf{TypeError}}                                 Traceback (most recent call last)
\textcolor{ansi-green-intense}{\textbf{<ipython-input-80-422fa5aaa16a>}} in \textcolor{ansi-cyan}{<module>}
\textcolor{ansi-green-intense}{\textbf{----> 1}}\textcolor{ansi-yellow-intense}{\textbf{ }}newdf\textcolor{ansi-yellow-intense}{\textbf{.}}loc\textcolor{ansi-yellow-intense}{\textbf{[}}\textcolor{ansi-yellow-intense}{\textbf{(}}newdf\textcolor{ansi-yellow-intense}{\textbf{[}}\textcolor{ansi-blue-intense}{\textbf{'a'}}\textcolor{ansi-yellow-intense}{\textbf{]}}\textcolor{ansi-yellow-intense}{\textbf{<}}\textcolor{ansi-cyan-intense}{\textbf{0.3}}\textcolor{ansi-yellow-intense}{\textbf{)}} \textcolor{ansi-green-intense}{\textbf{and}} \textcolor{ansi-yellow-intense}{\textbf{(}}newdf\textcolor{ansi-yellow-intense}{\textbf{[}}\textcolor{ansi-blue-intense}{\textbf{'b'}}\textcolor{ansi-yellow-intense}{\textbf{]}}\textcolor{ansi-yellow-intense}{\textbf{>}}\textcolor{ansi-cyan-intense}{\textbf{0.2}}\textcolor{ansi-yellow-intense}{\textbf{)}}\textcolor{ansi-yellow-intense}{\textbf{]}}\textcolor{ansi-red-intense}{\textbf{\#\#this function is use to give condition \#\#}}

\textcolor{ansi-green-intense}{\textbf{\textasciitilde{}\textbackslash{}anaconda\textbackslash{}lib\textbackslash{}site-packages\textbackslash{}pandas\textbackslash{}core\textbackslash{}ops\textbackslash{}common.py}} in \textcolor{ansi-cyan}{new\_method}\textcolor{ansi-blue-intense}{\textbf{(self, other)}}
\textcolor{ansi-green}{     63}         other \textcolor{ansi-yellow-intense}{\textbf{=}} item\_from\_zerodim\textcolor{ansi-yellow-intense}{\textbf{(}}other\textcolor{ansi-yellow-intense}{\textbf{)}}
\textcolor{ansi-green}{     64} 
\textcolor{ansi-green-intense}{\textbf{---> 65}}\textcolor{ansi-yellow-intense}{\textbf{         }}\textcolor{ansi-green-intense}{\textbf{return}} method\textcolor{ansi-yellow-intense}{\textbf{(}}self\textcolor{ansi-yellow-intense}{\textbf{,}} other\textcolor{ansi-yellow-intense}{\textbf{)}}
\textcolor{ansi-green}{     66} 
\textcolor{ansi-green}{     67}     \textcolor{ansi-green-intense}{\textbf{return}} new\_method

\textcolor{ansi-green-intense}{\textbf{\textasciitilde{}\textbackslash{}anaconda\textbackslash{}lib\textbackslash{}site-packages\textbackslash{}pandas\textbackslash{}core\textbackslash{}ops\textbackslash{}\_\_init\_\_.py}} in \textcolor{ansi-cyan}{wrapper}\textcolor{ansi-blue-intense}{\textbf{(self, other)}}
\textcolor{ansi-green}{    368}         rvalues \textcolor{ansi-yellow-intense}{\textbf{=}} extract\_array\textcolor{ansi-yellow-intense}{\textbf{(}}other\textcolor{ansi-yellow-intense}{\textbf{,}} extract\_numpy\textcolor{ansi-yellow-intense}{\textbf{=}}\textcolor{ansi-green-intense}{\textbf{True}}\textcolor{ansi-yellow-intense}{\textbf{)}}
\textcolor{ansi-green}{    369} 
\textcolor{ansi-green-intense}{\textbf{--> 370}}\textcolor{ansi-yellow-intense}{\textbf{         }}res\_values \textcolor{ansi-yellow-intense}{\textbf{=}} comparison\_op\textcolor{ansi-yellow-intense}{\textbf{(}}lvalues\textcolor{ansi-yellow-intense}{\textbf{,}} rvalues\textcolor{ansi-yellow-intense}{\textbf{,}} op\textcolor{ansi-yellow-intense}{\textbf{)}}
\textcolor{ansi-green}{    371} 
\textcolor{ansi-green}{    372}         \textcolor{ansi-green-intense}{\textbf{return}} self\textcolor{ansi-yellow-intense}{\textbf{.}}\_construct\_result\textcolor{ansi-yellow-intense}{\textbf{(}}res\_values\textcolor{ansi-yellow-intense}{\textbf{,}} name\textcolor{ansi-yellow-intense}{\textbf{=}}res\_name\textcolor{ansi-yellow-intense}{\textbf{)}}

\textcolor{ansi-green-intense}{\textbf{\textasciitilde{}\textbackslash{}anaconda\textbackslash{}lib\textbackslash{}site-packages\textbackslash{}pandas\textbackslash{}core\textbackslash{}ops\textbackslash{}array\_ops.py}} in \textcolor{ansi-cyan}{comparison\_op}\textcolor{ansi-blue-intense}{\textbf{(left, right, op)}}
\textcolor{ansi-green}{    242} 
\textcolor{ansi-green}{    243}     \textcolor{ansi-green-intense}{\textbf{elif}} is\_object\_dtype\textcolor{ansi-yellow-intense}{\textbf{(}}lvalues\textcolor{ansi-yellow-intense}{\textbf{.}}dtype\textcolor{ansi-yellow-intense}{\textbf{)}}\textcolor{ansi-yellow-intense}{\textbf{:}}
\textcolor{ansi-green-intense}{\textbf{--> 244}}\textcolor{ansi-yellow-intense}{\textbf{         }}res\_values \textcolor{ansi-yellow-intense}{\textbf{=}} comp\_method\_OBJECT\_ARRAY\textcolor{ansi-yellow-intense}{\textbf{(}}op\textcolor{ansi-yellow-intense}{\textbf{,}} lvalues\textcolor{ansi-yellow-intense}{\textbf{,}} rvalues\textcolor{ansi-yellow-intense}{\textbf{)}}
\textcolor{ansi-green}{    245} 
\textcolor{ansi-green}{    246}     \textcolor{ansi-green-intense}{\textbf{else}}\textcolor{ansi-yellow-intense}{\textbf{:}}

\textcolor{ansi-green-intense}{\textbf{\textasciitilde{}\textbackslash{}anaconda\textbackslash{}lib\textbackslash{}site-packages\textbackslash{}pandas\textbackslash{}core\textbackslash{}ops\textbackslash{}array\_ops.py}} in \textcolor{ansi-cyan}{comp\_method\_OBJECT\_ARRAY}\textcolor{ansi-blue-intense}{\textbf{(op, x, y)}}
\textcolor{ansi-green}{     54}         result \textcolor{ansi-yellow-intense}{\textbf{=}} libops\textcolor{ansi-yellow-intense}{\textbf{.}}vec\_compare\textcolor{ansi-yellow-intense}{\textbf{(}}x\textcolor{ansi-yellow-intense}{\textbf{.}}ravel\textcolor{ansi-yellow-intense}{\textbf{(}}\textcolor{ansi-yellow-intense}{\textbf{)}}\textcolor{ansi-yellow-intense}{\textbf{,}} y\textcolor{ansi-yellow-intense}{\textbf{.}}ravel\textcolor{ansi-yellow-intense}{\textbf{(}}\textcolor{ansi-yellow-intense}{\textbf{)}}\textcolor{ansi-yellow-intense}{\textbf{,}} op\textcolor{ansi-yellow-intense}{\textbf{)}}
\textcolor{ansi-green}{     55}     \textcolor{ansi-green-intense}{\textbf{else}}\textcolor{ansi-yellow-intense}{\textbf{:}}
\textcolor{ansi-green-intense}{\textbf{---> 56}}\textcolor{ansi-yellow-intense}{\textbf{         }}result \textcolor{ansi-yellow-intense}{\textbf{=}} libops\textcolor{ansi-yellow-intense}{\textbf{.}}scalar\_compare\textcolor{ansi-yellow-intense}{\textbf{(}}x\textcolor{ansi-yellow-intense}{\textbf{.}}ravel\textcolor{ansi-yellow-intense}{\textbf{(}}\textcolor{ansi-yellow-intense}{\textbf{)}}\textcolor{ansi-yellow-intense}{\textbf{,}} y\textcolor{ansi-yellow-intense}{\textbf{,}} op\textcolor{ansi-yellow-intense}{\textbf{)}}
\textcolor{ansi-green}{     57}     \textcolor{ansi-green-intense}{\textbf{return}} result\textcolor{ansi-yellow-intense}{\textbf{.}}reshape\textcolor{ansi-yellow-intense}{\textbf{(}}x\textcolor{ansi-yellow-intense}{\textbf{.}}shape\textcolor{ansi-yellow-intense}{\textbf{)}}
\textcolor{ansi-green}{     58} 

\textcolor{ansi-green-intense}{\textbf{pandas\textbackslash{}\_libs\textbackslash{}ops.pyx}} in \textcolor{ansi-cyan}{pandas.\_libs.ops.scalar\_compare}\textcolor{ansi-blue-intense}{\textbf{()}}

\textcolor{ansi-red-intense}{\textbf{TypeError}}: '<' not supported between instances of 'str' and 'float'
    \end{Verbatim}

    \begin{tcolorbox}[breakable, size=fbox, boxrule=1pt, pad at break*=1mm,colback=cellbackground, colframe=cellborder]
\prompt{In}{incolor}{81}{\boxspacing}
\begin{Verbatim}[commandchars=\\\{\}]
\PY{c+c1}{\PYZsh{}\PYZsh{} inplace=true is used to change in to dataframe \PYZsh{}}
\end{Verbatim}
\end{tcolorbox}

    \begin{tcolorbox}[breakable, size=fbox, boxrule=1pt, pad at break*=1mm,colback=cellbackground, colframe=cellborder]
\prompt{In}{incolor}{82}{\boxspacing}
\begin{Verbatim}[commandchars=\\\{\}]
\PY{n}{newdf}\PY{p}{[}\PY{l+s+s1}{\PYZsq{}}\PY{l+s+s1}{b}\PY{l+s+s1}{\PYZsq{}}\PY{p}{]}\PY{o}{.}\PY{n}{isnull}\PY{p}{(}\PY{p}{)}
\end{Verbatim}
\end{tcolorbox}

            \begin{tcolorbox}[breakable, size=fbox, boxrule=.5pt, pad at break*=1mm, opacityfill=0]
\prompt{Out}{outcolor}{82}{\boxspacing}
\begin{Verbatim}[commandchars=\\\{\}]
0      False
1      False
2      False
3      False
4      False
       {\ldots}
329    False
330    False
331    False
332    False
333    False
Name: b, Length: 334, dtype: bool
\end{Verbatim}
\end{tcolorbox}
        
    \begin{tcolorbox}[breakable, size=fbox, boxrule=1pt, pad at break*=1mm,colback=cellbackground, colframe=cellborder]
\prompt{In}{incolor}{87}{\boxspacing}
\begin{Verbatim}[commandchars=\\\{\}]
\PY{n}{newdf}\PY{o}{.}\PY{n}{loc}\PY{p}{[}\PY{p}{[}\PY{l+m+mi}{1}\PY{p}{,}\PY{l+m+mi}{2}\PY{p}{,}\PY{l+m+mi}{45}\PY{p}{,}\PY{l+m+mi}{68}\PY{p}{,}\PY{l+m+mi}{79}\PY{p}{,}\PY{l+m+mi}{89}\PY{p}{]}\PY{p}{,}\PY{p}{:}\PY{p}{]}
\end{Verbatim}
\end{tcolorbox}

            \begin{tcolorbox}[breakable, size=fbox, boxrule=.5pt, pad at break*=1mm, opacityfill=0]
\prompt{Out}{outcolor}{87}{\boxspacing}
\begin{Verbatim}[commandchars=\\\{\}]
           a         b         c         d         e
1   0.337682  0.032490  0.578410  0.196446  0.459194
2   0.999539  0.553314  0.687496  0.716025  0.397865
45  0.541209  0.917080  0.211663  0.263021  0.447161
68  0.838151  0.466380  0.120295  0.043531  0.776046
79  0.874318  0.439572  0.550404  0.131665  0.231317
89  0.342314  0.804204  0.169982  0.868972  0.864078
\end{Verbatim}
\end{tcolorbox}
        
    \begin{tcolorbox}[breakable, size=fbox, boxrule=1pt, pad at break*=1mm,colback=cellbackground, colframe=cellborder]
\prompt{In}{incolor}{ }{\boxspacing}
\begin{Verbatim}[commandchars=\\\{\}]

\end{Verbatim}
\end{tcolorbox}


    % Add a bibliography block to the postdoc
    
    
    
\end{document}
